\documentclass[UTF8]{ctexart}

\usepackage{anyfontsize}
\usepackage{geometry}
    \geometry{left=4cm,right=4cm,top=2cm,bottom=2cm}
\usepackage{amsmath, amssymb, amsthm}
    \newtheorem{definition}{Definition}
\usepackage{caption} 	 % 标题
\usepackage{xcolor} 	 % 颜色
\usepackage{graphicx} 	 % 引用图片
\usepackage{float}
\usepackage{framed} 	 % 方框 \begin{framed}
\usepackage{tcolorbox}
    \newtcolorbox{defbox}{colback = blue!20!green!10!white}
    \newtcolorbox{thmbox}{colback = black!5!white}
\usepackage{indentfirst} 	 % 首行缩进 
    \setlength{\parindent}{0em}
\usepackage{setspace} 	 % 行间距 \begin{spacing}{arg}
\usepackage{extarrows} 	 % 箭头宏包 \xLongrightarrow 
\usepackage{esvect} 	 % 向量箭头 \vv{}
\usepackage{siunitx} 	 % 国际单位 \si{unit} \SI{number}{unit} 
\usepackage{esint} 	 % 积分符号
\usepackage{mathrsfs}


% font
\newcommand{\ve}[1]{\boldsymbol{\mathbf{#1}}}
\newcommand{\unit}[1]{\boldsymbol{\mathbf{\hat{#1}}}}
\newcommand{\mr}{\mathrm}
\newcommand{\mcal}{\mathcal}
\newcommand{\mscr}{\mathscr}
% common symbol
\newcommand{\E}{\mathrm e}
\renewcommand{\I}{\mathrm i}
\newcommand{\R}{\mathbb R}
\newcommand{\Z}{\mathbb Z}
\newcommand{\N}{\mathbb N}
\newcommand{\Q}{\mathbb Q}
\newcommand{\C}{\mathbb C}
% differentiation
\def \DD #1.#2.#3 {\dfrac{d^{#1} #2}{d #3^{#1}}}
\def \PP #1.#2.#3 {\dfrac{\partial^{#1} #2}{\partial #3^{#1}}}
\def \dd #1.#2 {\dfrac{d #1}{d #2}}
\def \pp #1.#2 {\dfrac{\partial #1}{\partial #2}} 
% matrix
\newcommand{\transp}{^{\top}}
\DeclareMathOperator{\tr}{tr}

\pagestyle{empty}

\begin{document}
\section{Matrix}
\subsection{Permutation}
\begin{definition}
    A permutation
\end{definition}



\section{Group}
\begin{defbox}
    \begin{definition}
        A group $ (G, \cdot) $ is a set $ G $ equipped with a binary operation $ \cdot $ which follows four axioms, namely \textbf{closure}, \textbf{associativity}, \textbf{identity} and \textbf{invertibility}.
    \end{definition}    
\end{defbox}


The four axioms are defined below:
\paragraph{closure}
For all $ a, b $ in $ G $, the result of operation $ \cdot $ is still in $ G $. This can be written in the form: $ \forall a, b \in G, a \cdot b \in G $.

\paragraph{associativity}
$ \forall a, b, c \in G $, $ (a \cdot b) \cdot c = a \cdot (b \cdot c) $.

\paragraph{identity}
$ \exists\ e \in G $ such that, $ \forall a \in G $, the equation $ e \cdot a = a \cdot e = a $ holds. Such an element is unique and is called the \textbf{identity element}.

\paragraph{invertibility}
For each $ a \in G $, $ \exists\ b $ in $ G $, commonly denoted $ a^{-1} $, such that $ a \cdot b = b \cdot a = e $, where $ e $ is the identity element.



\end{document}