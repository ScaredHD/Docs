\documentclass[UTF8]{ctexart}

\usepackage{geometry}
    \geometry{left=4cm,right=4cm,top=2cm,bottom=2cm}


\usepackage{amsmath, amssymb, amsthm}
\theoremstyle{remark} 
\newtheorem{Definition}{\hspace{1em} 定义}[subsection]
\newtheorem{Theorem}{\hspace{1em} 定理}[subsection]
\newtheorem*{Property}{\hspace{1em} 性质}
\newtheorem{Concept}{\hspace{1em} }

\newenvironment{definition}[1]
{\begin{Definition}[#1] \hspace{0.5em}}
{\end{Definition} \vspace{0.5em}}

\newenvironment{theorem}[1]
{\begin{Theorem}[#1] \hspace{0.5em}}
{\end{Theorem} \vspace{0.5em}}

\newenvironment{property}
{\begin{Property} \hspace{0.5em}}
{\end{Property} \vspace{0.5em}}

\newenvironment{pf}
{\begin{proof} }
{\end{proof} \vspace{0.5em}}

\newenvironment{concept}[1]
{\hspace{1em} #1. \hspace{0.5em}}
{\vspace{0.5em}}



\usepackage{caption} 	 % 标题
\usepackage{xcolor} 	 % 颜色
\usepackage{graphicx} 	 % 引用图片
\usepackage{float}
\usepackage{multirow}
\usepackage{framed} 	 % 方框 \begin{framed}

\usepackage{indentfirst} 	 % 首行缩进 
    \setlength{\parindent}{0em}
\usepackage{setspace} 	 % 行间距 \begin{spacing}{arg}

\usepackage{extarrows} 	 % 箭头宏包 \vv{}
\usepackage{esvect} 	 %向量箭头
\usepackage[version=3]{mhchem} 	 %化学方程式 \ce{}
\usepackage{siunitx} 	 %国际单位 \si{unit} \SI{number}{unit} 


\newcommand{\df}{\dfrac}
\newcommand{\f}{\frac}
\newcommand{\mb}{\mathbf}
\newcommand{\e}{\mathrm e}
\newcommand{\ii}{\mathrm i}
\newcommand{\rev}{^{-1}}
\def \DD #1.#2.#3 {\dfrac{d^{#1} #2}{d #3^{#1}}}
\def \PP #1.#2.#3 {\dfrac{\partial^{#1} #2}{\partial #3^{#1}}}
\def \dd #1.#2 {\dfrac{d #1}{d #2}}
\def \pp #1.#2 {\dfrac{\partial #1}{\partial #2}} 

\pagestyle{empty}


\begin{document}
\section{行列式}
\subsection{行列式定义}
\begin{framed}
    \begin{definition}{\textit{行列式}}
        矩阵 $ \mb A $ 的行列式为一个确定的数, 记作 $ \det \mb A $ 或 $ |\mb A| $. 对 $ n \times n $ 矩阵 $ \mb A $, 其行列式可由下面方式定义:
        \[ \det \mb A = \sum_{\sigma}\left[ (-1)^{\tau(\sigma_1, \sigma_2, \dots, \sigma_n)} \prod_{1 \leqslant i \leqslant n} a_{i, \sigma_i} \right] \]

        其中, $ \sigma $ 为列角标的排列, $ \tau $ 为 $ \sigma $ 排列的逆序数.
    \end{definition}

    \begin{property} 
        \begin{itemize}
            \item 两行(列)\underline{\textbf{互换}}, 行列式变号.
            \item 一行(列)元素全为 $ 0 $, 行列式的值为 $ 0 $.
            \item 某一行(列)有\underline{\textbf{公因子}} $ k $, 则可以提出 $ k $.
            \item 某一行(列)的每个元素是两数之和, 则此行列式可拆分为两个相加的行列式.
            \item 有两行(列)对应\underline{\textbf{成比例}}或\underline{\textbf{相同}}, 则此行列式的值为 $ 0 $.
            \item 将一行(列)的常数\underline{\textbf{倍加}}进另一行(列)里, 行列式的值不变.
        \end{itemize}

        行列式转置,值不变: 
        \[ |\mb A| = |\mb A^{\rm T}| .\] 
    
        方块矩阵的乘积的行列式等于行列式的乘积:
        \[ \det (\mb{AB}) = \det \mb A \cdot \det \mb B \]

        $ \mb A $ 为 $ n $ 阶矩阵, $ c $ 为常数:
        \[ |c \mb A| = c^n |\mb A| \]

        \[ |\mb A^{-1}| = \dfrac{1}{|\mb A|} \]


    \end{property}
\end{framed}

\begin{framed}
    \begin{concept}{\textit{萨吕法则}}
        对于 $ 2 $ 或 $ 3 $ 阶行列式:
        \[ 
        \begin{vmatrix}
            {\color{red} a} & {\color{blue} b} \\
            {\color{blue} c} & {\color{red} d}
        \end{vmatrix} = {\color{red} ad} - {\color{blue} bc}    
        \]

        \begin{align*}
            \begin{vmatrix} 
                \square & \square & \square \\
                \square & \square & \square \\
                \square & \square & \square
            \end{vmatrix}
            &= 
            \begin{vmatrix} 
                \blacksquare & \square & \square \\
                \square & \blacksquare & \square \\
                \square & \square & \blacksquare
            \end{vmatrix} + 
            \begin{vmatrix} 
                \square & \blacksquare & \square \\
                \square & \square & \blacksquare \\
                \blacksquare & \square & \square
            \end{vmatrix} +
            \begin{vmatrix} 
                \square & \square & \blacksquare \\
                \blacksquare & \square & \square \\
                \square & \blacksquare & \square
            \end{vmatrix} \\
            &-
            \begin{vmatrix} 
                \square & \square & \blacksquare \\
                \square & \blacksquare & \square \\
                \blacksquare & \square & \square
            \end{vmatrix} -
            \begin{vmatrix} 
                \blacksquare & \square & \square \\
                \square & \square & \blacksquare \\
                \square & \blacksquare & \square
            \end{vmatrix} -
            \begin{vmatrix} 
                \square & \blacksquare & \square \\
                \blacksquare & \square & \square \\
                \square & \square & \blacksquare
            \end{vmatrix}
        \end{align*}     
    \end{concept}
    
\end{framed}

三角矩阵的行列式为主对角线上元素的乘积:
\[ 
\begin{vmatrix}
    {\color{red} a_{11}} & a_{12} & \cdots & a_{1n} \\
    0 & {\color{red} a_{22}} & \cdots & a_{2n} \\
    0 & 0 & \ddots & \vdots \\
    0 & 0 & \cdots & {\color{red} a_{nn}} \\
\end{vmatrix} = \prod_{i = 1}^n a_{ii}   
\]

使用上述性质, 应用初等变换将行列式化为三角形式, 便可求得其值.

次对角线上的三角矩阵行列式等于次对角线乘积乘 $ -1 $ 的 $ \dfrac{n(n-1)}{2} $ 次方:
\[
\begin{vmatrix}
    a_{11} & a_{12} & \cdots & {\color{red} a_{1n}} \\
    a_{21} & \cdots & {\color{red} a_{2, n-1}} &  \\
    \vdots & \cdots &  &  \\
    {\color{red} a_{n1}} &  &  &  \\
\end{vmatrix} = (-1)^{ n(n-1)/2}\prod_{i + j = n + 1} a_{ij}   
\]


\begin{framed}
    \begin{definition}{\textit{余子式}}
        将某个元素 $ a_{ij} $ 所在行和列删去, 剩余元素组成的行列式为 $ a_{ij} $ 的余子式, 记作 $ M_{ij} $. 例:

    \[ M_{23} = \begin{vmatrix}
        \blacksquare & \blacksquare & \square \\
        \square & \square & \square \\
        \blacksquare & \blacksquare & \square
        \end{vmatrix} = \begin{vmatrix}
        \blacksquare & \blacksquare \\
        \blacksquare & \blacksquare
    \end{vmatrix} \]
    \end{definition}

    \begin{definition}{\textit{代数余子式}}
    余子式前带上符号, 就成为代数余子式
        \[ C_{ij} = (-1)^{i + j} M_{ij} \]

        \[ \begin{bmatrix} 
            + & - & + & - & + \\
            - & + & - & + & - \\
            + & - & + & - & + \\
            - & + & - & + & - \\
            + & - & + & - & + 
        \end{bmatrix} \]
    \end{definition}

    \begin{definition}{\textit{余子矩阵}}
        若矩阵 $ \mb C $ 里每一个元素都是 $ \mb A $ 对应元素的代数余子式, 则 $ \mb C $ 称为 $ \mb A $ 的余子矩阵. 即:
        \[ \mb C = \begin{bmatrix}
            C_{11} & C_{12} & \cdots \\
            C_{21} & C_{22} & \cdots \\
            \vdots & \vdots & \ddots 
        \end{bmatrix} \]
    \end{definition}
\end{framed}

\begin{framed}
    \begin{theorem}{\textit{拉普拉斯展开}}
        矩阵 $ \mb A_{n \times n} $ 可按任意一行或列展开, 得到 $ n $ 个 $ (n-1) $ 阶方阵之和. 沿第 $ i $ 行展开的表达式:
        \[ \det \mb A = \sum_{k = 1}^{n} a_{ik} C_{ik} .\]

        其中, $ C_{ik} $ 为 $ a_{ik} $ 的代数余子式. 按列展开同理
        
    \end{theorem}    
\end{framed}

\begin{framed}
    \begin{definition}{\textit{范德蒙德行列式}}
        形如:
        \[ 
            V = \begin{vmatrix}
                1 & 1 & \cdots & 1 \\
                x_1 & x_2 & \cdots & x_n \\
                x_1^2 & x_2^2 & \cdots & x_n^2 \\ 
                \vdots & \vdots & \ddots & \vdots \\
                x_1^{n-1} & x_2^{n-1} & \cdots & x_n^{n-1}
            \end{vmatrix}
        \]
        
        的行列式称为范德蒙德行列式. 其值有:
        \[ V = \prod_{1 \leqslant j < i \leqslant n} (x_i - x_j) \]

    \end{definition}
\end{framed}




\subsection{行列式应用}
\begin{framed}
    \begin{definition}{\textit{伴随矩阵}}
        将 $ \mb A $ 的余子矩阵转置, 得到的矩阵为伴随矩阵, 记作: $ \operatorname{adj} \mb A $ 或 $ \mb A^* $. 即:
        \[ \mb A^* = \mb C^{\rm T} .\]
    \end{definition}
\end{framed}


\begin{framed}
    \begin{definition}{\textit{伴随矩阵与逆矩阵}}
        $ n $ 阶矩阵 $ \mb A $ 和其伴随矩阵 $ \mb A^* $ 有如下关系:
        \[ \mb A \mb A^* = |\mb A|\mb I_n .\]

        所以有:
        \[ \mb A^{-1} = \dfrac{\mb A^*}{|\mb A|} .\]
    \end{definition}
\end{framed}


\begin{framed}
    \begin{concept}{\textit{伴随矩阵与逆矩阵的行列式}}
    $ \mb A $ 为 $ n \times n $ 矩阵, 则:
    \[ |\mb A^*| = |\mb A|^{n - 1} \]
    \[ |\mb A^{-1}| = \left| \dfrac{\mb A^*}{|\mb A|} \right| = \dfrac{1}{|\mb A|^n} |\mb A^*| = \dfrac{|\mb A|^{n - 1}}{|\mb A|^n} = \dfrac{1}{|\mb A|} \]
    \end{concept}
\end{framed}




\end{document}

