\documentclass[UTF8]{ctexart}
\hfuzz=4pt

\usepackage{parskip}
    \setlength{\parindent}{0em}
\usepackage{geometry}
    \geometry{left=4cm,right=4cm,top=2cm,bottom=2cm}
\usepackage{amsmath, amssymb, amsthm, mathtools}
\usepackage{thmtools}
    \renewcommand\qedsymbol{$\blacksquare$}
    \newtheoremstyle{mystyle}{1em plus .2 em minus .2em}{1em plus .2 em minus .2em}{}{}{\bfseries}{.}{.5em}{}
    \theoremstyle{mystyle}
    \declaretheorem[numberwithin=section]{definition}
    \declaretheorem[numberwithin=section]{theorem}
    \declaretheorem[numberwithin=section]{proposition}
\usepackage{caption}
\usepackage{xcolor}
\usepackage{graphicx}
\usepackage{float}
\usepackage{setspace} 	 % 行间距 \begin{spacing}{arg}
\usepackage{extarrows}
\usepackage{esint}
\usepackage{hyperref}
    \hypersetup{colorlinks=true,linktoc=all,linkcolor=blue}

\newcommand{\ve}[1]{\boldsymbol{\mathbf{#1}}}
\newcommand{\unit}[1]{\boldsymbol{\mathbf{\hat{#1}}}}
\renewcommand{\r}{\mathrm}
\renewcommand{\cal}{\mathcal}
\newcommand{\scr}{\mathscr}
\newcommand{\E}{\mathrm e}
\renewcommand{\I}{\mathrm i}
\newcommand{\R}{\mathbb R}
\newcommand{\Z}{\mathbb Z}
\newcommand{\N}{\mathbb N}
\newcommand{\Q}{\mathbb Q}
\renewcommand{\C}{\mathbb C}
\DeclarePairedDelimiter\set{\lbrace}{\rbrace}
\DeclarePairedDelimiter\norm{\lVert}{\rVert}
\def \DD #1.#2.#3 {\dfrac{d^{#1} #2}{d #3^{#1}}}
\def \PP #1.#2.#3 {\dfrac{\partial^{#1} #2}{\partial #3^{#1}}}
\def \dd #1.#2 {\dfrac{d #1}{d #2}}
\def \pp #1.#2 {\dfrac{\partial #1}{\partial #2}} 
\newcommand{\del}{\nabla}
\renewcommand{\epsilon}{\varepsilon}
\DeclareMathOperator{\interior}{int}

\pagestyle{empty}

\begin{document}
\section{集合的确界}
\begin{definition}[\text{上确界/最小上界}]
    对于集合 $ S \subseteq \R $ 和 $ \beta \in \R $, 称 $ \beta $ 为 $ S $ 的上确界, 记作 $ \sup S = \beta $, 当且仅当:
    \begin{enumerate}
        \item $ \forall x \in S $, $ x \leqslant \beta $
        \item $ \forall \epsilon > 0 $, 能够找到 $ x \in S $, $ \beta - \epsilon < x $
    \end{enumerate}
    第一条说明 $ \beta $ 为上界, 第二条说明 $ \beta $ 是最小的上界.
\end{definition}

根据上确界的定义, 容易导出下面的常用性质:
\begin{enumerate}
    \item (确界的最小性) 若 $ M $ 为 $ S $ 的上界, 则 $ \sup S \leqslant M $
    \item (逼近性质) 对任意 $ a < \beta $, 存在 $ x \in S $, $ a < x \leqslant \beta $
    \item 若 $ S $ 存在最大值 $ m $, 则 $ S $ 的上确界 $ \sup S = m $.
\end{enumerate}

同理可以定义下确界/最大下界, 记作 $ \inf S $.

\subsection{确界的性质}
对于集合 $ A $ 和 $ B $, 定义两种运算 $ A + B $ 和 $ A \cdot B $:
\[ A + B \coloneqq \set{a + b \mid a \in A, b \in B} \,,\]
\[ A \cdot B \coloneqq \set{ab \mid a \in A, b \in B} \,.\]

\paragraph{例}
$ \set{1, 3} + \set{5, 9} = \set{6, 10, 8, 12} $, \\
$ \set{2, -2} + (-1, 1) = (-3, -1) \cup (1, 3) $; \\
$ \set{1, 3} \cdot \set{5, 9} = \set{5, 9, 15, 27} $, \\
$ (1, 2) \cdot (3, 4) = (3, 8) $.

\begin{proposition}
    对于非空集合 $ A $, $ B $:
    \[ \sup(A + B) = \sup A + \sup B \,,\]
    若 $ A $, $ B $ 是正实数的集合:
    \[ \sup(A \cdot B) = \sup A \cdot \sup B \,. \]
\end{proposition}

\begin{proof}[\text{证明可加性}]
    记 $ \sup A = \alpha $, $ \sup B = \beta $, $ A \cdot B = C $. 显然, $ \alpha + \beta $ 是 $ C $ 的上界. 只需证明对于任意正实数 $ \epsilon $, $ \alpha + \beta - \epsilon < a_0 + b_0 $, 其中 $ a_0 $, $ b_0 $ 是 $ A $, $ B $ 中的元素, $ a_0 + b_0 $ 是 $ C $ 的元素. 由于 $ \alpha $, $ \beta $ 分别为 $ A $, $ B $ 的上确界, 按照定义, 对于任意正整数 $ \epsilon' $ 有:
    \[ \alpha - \epsilon' < a_0 \,,\]
    \[ \beta - \epsilon' < b_0 \,.\]
    两边同时相加:
    \[ \alpha + \beta - 2\epsilon' < a_0 + b_0 \,.\]
    而 $ 2\epsilon' $ 为任意正实数, 命题得证.
\end{proof}

\begin{proof}[\text{证明可乘性}]
    记 $ \sup A = \alpha $, $ \sup B = \beta $, $ A \cdot B = C $. 显然, $ \alpha \beta $ 是 $ C $ 的上界, 只需证明对于任意正实数 $ \epsilon $, $ \alpha \beta - \epsilon < a_0 b_0 $, 其中 $ a_0 $, $ b_0 $ 是 $ A $, $ B $ 中的元素, $ a_0 b_0 $ 是 $ C $ 的元素. 由于 $ \alpha $, $ \beta $ 分别为 $ A $, $ B $ 的上确界, 按照定义, 对于任意正实数 $ \epsilon' $ 和充分大的 $ n $, 有:
    \[ 0 < \alpha - \dfrac{\epsilon'}{n} < a_0 \,,\]
    \[ 0 < \beta - \dfrac{\epsilon'}{n} < b_0 \,.\]
    将不等式相乘:
    \begin{align*}
        \left( \alpha - \dfrac{\epsilon'}{n} \right) \left( \beta - \dfrac{\epsilon'}{n} \right) &< a_0 b_0 \\
        \alpha \beta - \dfrac{\epsilon'}{n}(\alpha + \beta) + \left( \dfrac{\epsilon'}{n} \right)^2 &< a_0 b_0 \\
        \alpha \beta - \dfrac{\epsilon'}{n} (\alpha + \beta) &< a_0 b_0
    \end{align*}
    由于 $ n $ 充分大, 可以取 $ n > \alpha + \beta $, 所以有:
    \[ \alpha \beta - \epsilon' < a_0 b_0 \,.\]
    这便证明了 $ \alpha \beta = \sup C $.
\end{proof}


\section{集合的拓扑性质}
几个定义 ($ S \subseteq \R^n $, $ \ve x \in \R^n $):

\paragraph{附着点 (Adherent point)}
$ \forall B(\ve x) \cap S \neq \varnothing $

$ S $ 所有的附着点称为 $ S $ 的闭包, 记作 $ \overline S $.

\paragraph{聚点 (Accumulation point)}
$ \forall B(\ve x) \cap (S \setminus \set{\ve x}) \neq \varnothing $

(1) $ A \cap (B \setminus C) = (A \setminus C) \cap B $; 

(2) $ S $ 的所有聚点的集合称为 $ S $ 的导集, 记作 $ S' $.

\paragraph{孤点 (Isolated point)}
$ \ve x $ 是附着点但不是聚点 



\subsection{开集与闭集}
$ \R^n $ 中的集合 $ S $ 为开集的充要条件有两个:
\begin{enumerate}
    \item $ \R^n \setminus S $ 为闭集
    \item $ \interior S = S $
\end{enumerate}

对于 $ \R^n $ 中的集合 $ S $, 若 $ S $ 是闭集, 则其和下面的命题是等价的:
\begin{enumerate}
    \item $ \R^n \setminus S $ 为开集
    \item $ S $ 包含所有附着点: $ \overline S \subseteq S $
    \item $ S $ 包含所有的聚点: $ S' \subseteq S $
    \item $ \overline S = S $
\end{enumerate}

上面的概念, 以及下面的性质, 可以从 $ \R^n $ 拓广到度量空间.

注意下面几个事实:
\begin{itemize}
    \item 集合开闭是独立的概念, 一个集合可以同时为开集和闭集, 也可以同时非开非闭
    \item $ \R^n $ 和 $ \varnothing $ 同时为开集和闭集
    \item $ S $ 的附着点可以被划分为聚点和孤点. 而孤点必然是在 $ S $ 中的, 也即如果 $ x \in \overline S $ 但 $ x \notin S $, $ x $ 只能为聚点
    \item $ S $ 内的点天然附着于 $ S $, 所以任何孤点都是该集合的附着点. 综合上一条, 此时 $ S $ 包含所有附着点和 $ S $ 包含所有聚点就是等价的描述
\end{itemize}

\subsection{几个实例}
(1) 对于区间 $ (a, b) $, $ (a, b] $, $ [a, b) $ 和 $ [a, b] $: 其区间端点 $ a $, $ b $ 为区间的聚点. 这四个区间的导集为 $ [a, b] $.

(2) $ S $ 的上确界是 $ S $ 的附着点: 设 $ \sup S = \beta $, 于是存在 $ x \in S $, $ \beta - \epsilon < x $, 另有 $ x \leqslant \beta < \beta + \epsilon $. 故 $ \beta - \epsilon < x < \beta + \epsilon $, 即 $ x \in B(\beta; \epsilon) $, 而 $ \epsilon $ 是任意正实数. 这说明 $ \beta $ 的任意领域 $ B(\beta) $ 里都包含 $ x $, 所以为附着点.

(3) 对于集合 $ S = \set*{\dfrac{1}{n} \colon n \in \Z^+} $. $ 1 $ 为 $ S $ 的最大值, 所以 $ \sup S = 1 $. 容易证明, $ \inf S = 0 $. 故 $ 0 $ 为附着点, 还可以证明 $ 0 $ 为 $ S $ 唯一的聚点, 然而 $ 0 \notin S $, 所以 $ S $ 不是闭集. 下面考虑是否为开集. 注意到点 $ 1 $ 的任意邻域 $ B(1; r) $ 都不包含于 $ S $, 所以 $ 1 $ 不是内点, $ S \neq \interior S $, $ S $ 也不是开集.

(4) $ S = \set*{\dfrac{1}{n} + \dfrac{1}{m} \colon m, n \in \Z^+} $. 下面的性质十分有用: $ (A \cup B)' = A' \cup B' $.

可以将 $ S $ 写作下面的形式:
\[ \set*{1 + \dfrac{1}{m}}_{m = 1}^\infty \cup \set*{\dfrac{1}{2} + \dfrac{1}{m}}_{m = 1}^\infty \cup \set*{\dfrac{1}{3} + \dfrac{1}{m}}_{m = 1}^\infty \cup \cdots \,.\]

所以可以得到: $ S' = \set*{1, \dfrac{1}{2}, \dfrac{1}{3}, \dots} \cup \set{0} $. 所以 $ S' \not\subseteq S $, $ S $ 不是闭集. 注意到所有的 $ B(2) $ 都不包含在 $ S $ 中, 所以 $ 2 $ 不是内点, $ S $ 也不是开集.


\section{极限与连续性}
\subsection{点列的极限}
\begin{definition}
    度量空间 $ (S, d) $ 中的点列 $ (x_n) $ 收敛于 $ p $, 当且仅当对于任意 $ \epsilon > 0 $, 可以找到 $ N > 0 $, 使得所有 $ n \geqslant N $ 时有 $ d(x_n, p) < \epsilon $. 记作: \[ \lim x_n = p \,.\]也可以写作: 当 $ n \to \infty $ 时, $ x_n \to p $.
\end{definition}

\paragraph{注意}
$ d_S (x_n, p) < \epsilon $ 等价于 $ d_\R \big( d_S(x_n, p), 0 \big) = |d_S(x_n, p)- 0| < \epsilon $, 此处 $ d_S(x_n, p) \in \R $. 也就是说 $ (S, d_S) $ 中的点列 $ x_n \to p $, 等价于其距离 $ d_S(x_n, p) $ 在 $ \R $ 中 $ d_S(x_n p) \to 0 $.

\subsection{附着点/聚点与点列的联系}
考虑度量空间 $ (M, d) $ 中的子集 $ S $ 和点 $ p $. 如果 $ p $ 是 $ S $ 的附着点, 意味着每一个邻域 $ B(p) $ 都包含 $ S $ 中的点. 那么对于每一个正整数 $ n = 1, 2, \dots $, 都能找到 $ S $ 中的点 $ x_n $ 满足 $ d(x_n, p) < 1 / n $. 当 $ n \to \infty $ 时, $ 1/n \to 0 $, 所以 $ d(x_n, p) \to 0 $, 意味着 $ x_n \to p $. 这样就找到了 $ S $ 中收敛于 $ p $ 的点列.

反过来, 如果 $ S $ 中存在序列 $ (x_n) $ 收敛于 $ p $, 按照定义, 对于任意 $ \epsilon > 0 $ , 取充分大的 $ n $ 总有 $ d(x_n, p) < \epsilon $. 即任意 $ B(p; \epsilon) $ 都包含 $ S $ 中的点. 于是 $ p $ 为 $ S $ 的附着点.

所以有下面的命题:
\begin{proposition}
    $ p $ 是 $ S $ 的附着点, 当且仅当 $ S $ 中存在收敛到 $ p $ 的点列. 
\end{proposition}

同时按照聚点的定义, 也有:
\begin{proposition}
    $ p $ 是 $ S $ 的聚点, 当且仅当 $ S - \set{p} $ 中存在收敛到 $ p $ 的点列.
\end{proposition}

由于附着点可以分为聚点和孤点两类. 下面分别讨论其性质.

如果 $ p $ 是 $ S $ 的孤点, $ S $ 中一定存在点列 $ (x_n) $ 收敛到 $ p $, 记其值域 $ T = \set{x_1, x_2, \dots} $. 按照孤点的定义: 存在 $ r > 0 $, $ B(p; r) $ 中只有 $ p $ 一个 $ S $ 中的点. 而序列收敛到 $ p $, 对于任意 $ \epsilon > 0 $, 都能找到 $ N > 0 $, 对 $ n \geqslant N $ 有 $ d(x_n, p) < \epsilon $, 即 $ x_n \in B(p; \epsilon) $. 那么取 $ \epsilon = r $, 意味着所有 $ n \geqslant N $, $ x_n = p $. 这说明 $ T $ 是一个有限集. 

收敛到孤点的序列值域有限, 但反过来, 收敛到 $ p $ 的序列 $ (x_n) $ 值域有限, 并不代表着 $ p $ 为孤点. 因为对于任意 $ p \in S $, 都存在常序列 $ p, p, p, \dots $ 收敛到 $ p $, 而 $ p $ 显然不一定为孤点.

所以, 如果一个收敛到 $ p $ 的序列值域为无穷集, 则 $ p $ 不可能为孤点, 于是只可能为聚点.

综上所述: 收敛序列值域无穷 $ \Longrightarrow $ 收敛到聚点; 收敛到孤点 $ \Longrightarrow $ 收敛序列值域有限.

\subsection{函数的极限}
考虑两个度量空间 $ (S, d_S) $ 和 $ (T, d_T) $, $ A $ 为 $ S $ 的子集, 设函数 $ f \colon A \to T $ 为函数.

\begin{definition}
    设 $ p $ 为 $ A $ 的聚点, $ b \in T $, 当对于任意 $ \epsilon > 0 $ 都存在 $ \delta > 0 $, 使得
    \[ 
        \text{当 } x \in A , 0 < d_S(x, p) < \delta \text{ 时: }
        d_T \big( f(x), b \big) < \epsilon \,,
    \] 
    则称 $ f(x) $ 在 $ p $ 处的极限为 $ b $, 记作:
    \[ 
        \lim_{x \to p} f(x) = b \,.
    \]
    或记作: $ x \to p $, $ f(x) \to b $.
\end{definition}

从邻域的角度阐述: 无论 $ B_T(b; \epsilon) $ 多么小, 总能找到 $ A $ 中的去心邻域 $ B_S(p; \delta) - \set{p} $, 使得其中 $ x $ 被映射到 $ B_T(b; \epsilon) $ 中.

另一种阐述方式: 无论 $ B_T(b; \epsilon) $ 多么小, 总能找到 $ A $ 中的去心邻域 $ B_A^0(p; \delta) = B_S(p; \delta) \cap A - \set{p} $, 使得其像 $ f(B_A^0(p; \delta)) \subseteq B_T(b; \epsilon) $.

需要注意的条件: 我们要求 $ A - \set{p} $ 中有点充分接近 $ p $, 所以 $ p $ 一定是定义域 $ A $ 的聚点. 如果为孤点的话, $ p $ 的去心 $ \delta $-邻域 $ B(p; \delta) - \set{p} $ 很有可能为空集. 那么将这个集合内的点映射到任何一个 $ B(b; \epsilon) $ 邻域都是空真的, 也就是说此时可以称 $ p $ 点的极限为任意 $ b $, 显然没有意义.

函数的极限和序列的极限关系如下:
\begin{proposition}
    $ \lim\limits_{x \to p} f(x) = b $ 当且仅当 $ A - \set{p} $ 内每一个收敛于 $ p $ 的点列 $ (x_n) $ 都有 $ \lim\limits_{n \to \infty} f(x_n) = b $.
\end{proposition}

\begin{proof} \ 

    正推: 如果 $ \lim\limits_{x \to p} f(x) = b $, 对于任意 $ \epsilon > 0 $, 都能找到 $ \delta > 0 $, 当 $ 0 < d(x, p) < \delta $ 时, $ d(f(x), b) < \epsilon $. 设 $ A - \set{p} $ 有点列 $ (x_n) $ 收敛于 $ p $, 可以找到 $ N $, 当 $ n \geqslant N $ 时, $ d(x_n, p) < \delta $, 此时有 $ d(f(x_n), b) < \epsilon $. 所以序列 $ f(x_n) \to b $.

    反推: 假设任意收敛于 $ p $ 的点列 $ (x_n) $ 都有 $ f(x_n) \to b $, 但 $ f(x) $ 不收敛到 $ b $. 说明存在 $ \epsilon > 0 $, 此时任意 $ \delta > 0 $, $ 0 < d(x, p) < \delta $ 内的 $ x $ 都有 $ d(f(x), b) \geqslant \epsilon $. 那么取 $ \delta = 1, 1/2, 1/3, \dots $ 可以得到对应的点列 $ x_1, x_2, \dots $ 此时 $ d(f(x_i), b) \geqslant \epsilon $. 点列 $ (x_i)_{i = 1}^{\infty} $ 收敛到 $ p $, 但 $ d(f(x_i), b) \geqslant \epsilon $, 所以 $ f(x_i) $ 不收敛到 $ b $, 这就产生了矛盾.
\end{proof}

\subsection{连续性}
\begin{definition}
    设 $ f \colon S \to T $ 为函数, $ p $ 为 $ S $ 内一点. 称 $ f $ 在 $ p $ 点连续, 当且仅当对于任意 $ \epsilon > 0 $, 都有 $ \delta > 0 $ 使得当 $ d_S(x, p) < \delta $ 时 $ d_T(f(x), f(p)) < \epsilon $.
\end{definition}

如果 $ p $ 为孤点, 很明显 $ f $ 在 $ p $ 处连续. 当 $ p $ 为 $ S $ 的聚点, 则当 $ x \to p $ 时 $ f(x) \to f(p) $.

\begin{proposition}
    设 $ f \colon S \to T $ 为函数, $ p \in S $, 则 $ f $ 在 $ p $ 处连续当且仅当 $ S $ 内每一个收敛到 $ p $ 的序列 $ (x_n) $ 都有 $ T $ 中的序列 $ (f(x_n)) $ 收敛到 $ f(p) $, 即:
    \[ 
        \lim_{n \to \infty} f(x_n) = f \big( \lim_{n \to \infty} x_n \big) \,.
    \]
\end{proposition}

\subsection{连续映射}
\subsubsection{映射}
\begin{definition}[\text{逆象}]
    函数 $ f \colon S \to T $, $ Y \subseteq T $, $ Y $ 的逆象定义如下: \[ f^{-1}(Y) \coloneqq \set{x \in S \colon f(x) \in Y} \,.\] 即定义域中所有映射到 $ Y $ 的元素集合.
\end{definition}

像和逆像的常用性质:
\begin{proposition}
    $ f \colon S \to T $ 为函数
    \begin{enumerate}
        \item 若 $ A \subseteq B $, $ f(A) \subseteq f(B) $, $ f^{-1}(A) \subseteq f^{-1}(B) $
        \item $ f(A \cup B) = f(A) \cup f(B) $, $ f(A \cap B) \subseteq f(A) \cap f(B) $
        \item $ f^{-1}(A \cup B) = f^{-1}(A) \cup f^{-1}(B) $, $ f^{-1}(A \cap B) = f^{-1}(A) \cap f^{-1}(B) $
    \end{enumerate}
\end{proposition}

注意, $ f(A \cap B) \subseteq f(A) \cap f(B) $, 此处不一定取等. 不妨取两个互斥的集合 $ A \cap B = \varnothing $, 此时 $ f(A \cap B) = \varnothing $, 而令一侧 $ f(A) \cap f(B) $ 可以非空, 只要 $ f $ 不是单射.

\begin{proposition}
    对于 $ f \colon S \to T $ 和 $ f^{-1} $. 设 $ Y \subseteq T $, 则有 $ f \big( f^{-1}(Y) \big) \subseteq Y $. 当且仅当 $ f $ 为满射时取得等号.
\end{proposition}

\begin{proof}
    按照定义即可. 考虑任意 $ y \in f( f^{-1}(Y) ) $, 一定存在 $ x \in f^{-1}(Y) $, $ y = f(x) $. 而 $ x \in f^{-1}(Y) $ 意味着 $ f(x) \in Y $, 所以 $ y \in Y $.
\end{proof}

\begin{proposition}
    对于 $ f \colon S \to T $ 和 $ f^{-1} $. 设 $ X \subseteq S $, 则有 $ X \subseteq f^{-1} \big( f (X) \big) $. 上式取等当且仅当其对所有 $ X \subseteq S $
\end{proposition}

\begin{proof}
    对于任意 $ x \in X $, $ f(x) \in f(X) $. 也就是说 $ x $ 映射到 $ f(X) $ 中, 也就一定在 $ f^{-1}(f(X)) = \set{z \in X \colon f(z) \in f(X)} $ 中.
\end{proof}

另一个有趣的等式可以由此导出:
\begin{proposition} \

    \begin{itemize}
        \item $ f^{-1}(f(f^{-1}(Y))) = f^{-1}(Y) $
        \item $ f(f^{-1}(f(X))) = f(X) $
    \end{itemize}
\end{proposition}

\begin{proof}
    对于第一个等式, 把最内层括号里 $ f^{-1}(Y) $ 看成整体, 可以得到 \[ f^{-1}(Y) \subseteq f^{-1}(f(f^{-1}(Y))) \,;\] 而对于 $ f(f^{-1}(Y)) $, 已知 $ f(f^{-1}(Y)) \subseteq Y $, 两边应用 $ f^{-1} $, 有 \[ f^{-1}(f(f^{-1}(Y))) \subseteq f^{-1}(Y) \,.\] 综上所述: $ f^{-1}(f(f^{-1}(Y))) = f^{-1}(Y) $.

    对于第二个等式, 同理.
\end{proof}


\subsubsection{连续映射}
连续映射满足一定性质: 陪域中开集的逆像仍为开集, 闭集的逆像仍为闭集.

\begin{proposition}
    函数 $ f \colon S \to T $ 是函数, 如果 $ f $ 在开集 $ U \subseteq T $ 上连续, 则其逆像 $ f^{-1}(U) $ 也是开集. 如果 $ f $ 在闭集 $ V \subseteq T $ 上连续, 则其逆像 $ f^{-1}(V) $ 也是闭集.
\end{proposition}

但是正向地看, 定义域里面的开集, 经过连续映射不一定为开集, 反例可以举常值函数. 定义域里的闭集, 经过连续映射不一定为闭集, 反例为 $ \arctan(\R) = (-\pi/2, \pi/2) $.

但是对于紧集 (闭且有界), 却能在正向映射时保持紧性.

\begin{proposition}
    如果函数 $ f \colon S \to T $ 在 $ X \subseteq S $ 上连续, 且 $ X $ 为 $ S $ 中的紧集, 则 $ f(X) $ 是 $ T $ 中的紧集.
\end{proposition}

证明需要用到 Heine-Borel 定理.

\begin{definition}
    对于欧氏空间中的函数 $ f \colon S \to \R^n $, 称 $ f $ 在 $ S $ 上有界, 当且仅当存在正数 $ M $, 使得所有 $ x \in S $ 都有 $ \norm{\ve f(x)} \leqslant M $.
\end{definition}

\begin{theorem}
    设 $ \ve f \colon S \to \R^n $ 在紧集 $ X \subseteq S $ 上连续, 则 $ \ve f $ 在 $ X $ 上有界.
\end{theorem}

上述定理反应了欧氏空间中函数和紧集的关系: 紧集经过连续映射一定是有界的.


\end{document}