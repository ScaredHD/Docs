\documentclass[UTF8]{ctexart}
\hfuzz=4pt

\usepackage{parskip}
    \setlength{\parindent}{0em}
\usepackage{geometry}
    \geometry{left=4cm,right=4cm,top=2cm,bottom=2cm}
\usepackage{amsmath, amssymb, amsthm, mathtools}
\usepackage{thmtools}
    \renewcommand\qedsymbol{$\blacksquare$}
    \declaretheorem[numberwithin=section,shaded={rulecolor=cyan,rulewidth=2pt,bgcolor=white}]{definition}
    \declaretheorem[numberwithin=section,shaded={rulecolor=orange,rulewidth=2pt, bgcolor=white}]{theorem}
\usepackage{caption}
\usepackage{xcolor}
\usepackage{graphicx}
\usepackage{float}
\usepackage{setspace} 	 % 行间距 \begin{spacing}{arg}
\usepackage{extarrows}
\usepackage{esint}
\usepackage{hyperref}
    \hypersetup{colorlinks=true,linktoc=all,linkcolor=blue}

\pagestyle{empty}

\renewcommand{\vec}[1]{\boldsymbol{\mathbf{#1}}}
\newcommand{\unit}[1]{\boldsymbol{\mathbf{\hat{#1}}}}
\renewcommand{\rm}{\mathrm}
\renewcommand{\cal}{\mathcal}
\newcommand{\scr}{\mathscr}
\newcommand{\E}{\mathrm e}
\renewcommand{\I}{\mathrm i}
\newcommand{\R}{\mathbb R}
\newcommand{\Z}{\mathbb Z}
\newcommand{\N}{\mathbb N}
\DeclarePairedDelimiter\set{\lbrace}{\rbrace}
\DeclarePairedDelimiter\norm{\lVert}{\rVert}

\title{复合函数极限}
\date{}

\begin{document}
\maketitle
\vspace{-4em}

考虑 $ g \colon A \to B $, $ f \colon B \to C $, $ \operatorname{range}(g) \subseteq B = \operatorname{dom}(f) $, 意味着 $ f \circ g $ 是有意义的. 若 $ \lim\limits_{x \to a} g(x) = b $, $ \lim\limits_{y \to b} f(y) = L $, 在满足下列任一条件时:
\begin{enumerate}
	\item $ f $ 在 $ b $ 连续
	\item $ g $ 在定义域内 $ a $ 附近 (不包括 $ a $) 取不到极限值 $ b $
	\item $ b = \infty $
\end{enumerate}

复合函数的极限存在且:
\[
	\lim_{x \to a} f \big( g(x) \big) = L \,.
\]

\paragraph{注}
从证明过程中可看出, 除了两极限都取最弱条件时 $ \lim g(x) = \pm\infty $, $ \lim\limits_{y \to + \infty} f(y) = L $ 无法进行复合, 其余三种情况均可按照上述方式复合: 
\begin{itemize}
	\item (弱, 强) $ \lim g(x) = \pm \infty $, $ \lim\limits_{y \to \pm \infty} f(y) = L $
	\item (强, 弱) $ \lim g(x) = + \infty $, $ \lim\limits_{y \to + \infty} f(y) = L $
	\item (强, 强) $ \lim g(x) = + \infty $, $ \lim\limits_{y \to \pm \infty} f(y) = L $
\end{itemize}

\begin{proof}
	(1) $ \lim\limits_{y \to b} f(x) = L $ 意味着 $ \forall \varepsilon $ $ \exists \delta $, $ (\forall y \in B \colon 0 < d(y, b) < \delta) $, 有 $ d \big( f(x), L \big) < \varepsilon $.

	(2) $ \lim\limits_{x \to a} g(x) = b $ 则对于 (1) 中的 $ \delta $, $ \exists \delta' $, $ ( \forall x \in A \colon 0 < d(x, a) < \delta' $, 有 $ d \big( g(x), b \big) < \delta $.

	要将 (2) 和 (1) 连接起来, 矛盾在于 (2) 中 $ d \big( g(x), b \big) < \delta $ 不是去心邻域, 而 (1) 中 $ 0 < d(y, b) < \delta $ 要求去心邻域.

	(第一种条件) 若 $ f $ 在 $ b $ 连续, $ \lim\limits_{y \to b} f(y) = f(b) = L $, 意味着 (1) 中 $ 0 < d(y, b) < \delta $ 的条件可以改写为 $ d(y, b) < \delta $, 而已经有 $ d \big( g(x), b \big) < \delta $, 于是 $ d \Big( f \big( g(x) \big), L \Big) < \varepsilon $.

	(第二种条件) 若 $ g $ 在 $ a $ 的一个去心邻域内取不到 $ b $, 故 (2) 中的 $ d \big( g(x), b \big) < \delta $ 可以变为 $ 0 < d \big( g(x), b \big) < \delta $, 由 $ (1) $, $ d \Big( f \big( g(x) \big), L \Big) < \epsilon $.

	(第三种条件) 若 $ b = \infty $, 根据定义, (2) 中最后为 $ |g(x)| > \delta $, (1) 中有对应的条件 $ |y| > \delta $. 故无需其它条件.
\end{proof}

前两种条件下的证明可以直接推广到任意度量空间, 而第三个条件 ($ b $ 为无穷大) 只适用于 $ \R $.



\end{document}