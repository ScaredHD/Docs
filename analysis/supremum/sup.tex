\documentclass{article}

\usepackage{parskip}
    \setlength{\parindent}{0em}
\usepackage{geometry}
    \geometry{left=4cm,right=4cm,top=2cm,bottom=2cm}
\usepackage{amsmath, amssymb, amsthm, mathtools}
\usepackage{thmtools}
    \renewcommand*{\proofname}{Proof}
    \renewcommand\qedsymbol{$\blacksquare$}
    \declaretheorem[numberwithin=section]{theorem}
    \declaretheorem[numberwithin=section]{proposition}
    \declaretheorem[numberwithin=section]{definition}
    \declaretheorem[numbered=no]{example}
\usepackage{esint}
\usepackage{hyperref}
    \hypersetup{colorlinks=true,linktoc=all,linkcolor=blue}
\usepackage[italic=true]{derivative}

\newcommand{\R}{\mathbb R}
\DeclarePairedDelimiter\set{\lbrace}{\rbrace}


\begin{document}
\section{Supremum}
\subsection{Sup and arithmetic operations}
\begin{definition}[Operations on sets]
    Given two sets $ A $ and $ B $ of real numbers, and $ r \in \R $, define following elementwise operations:
    \begin{itemize}
        \item Minkowski addition: $ A + B \coloneq \set{a + b \colon a \in A, b \in B} $
        \item Scalar addition: $ A + c \coloneq \set{a + c \colon a \in A} $ for some $ c \in \R $
        \item Scalar multiplication: $ cA = \set{ca \colon a \in A} $ for some $ c \in \R $
    \end{itemize}
\end{definition}  

\begin{theorem}
    For operations defined above, we have:
    \begin{itemize}
        \item $ \sup (A + B) = \sup A + \sup B $
        \item $ \sup (A + c) = \sup A + c $
        \item $ \sup (cA) = r \sup A $ for $ c \geqslant 0 $
    \end{itemize}
\end{theorem}

$ A + c $ can be seen as a special case $ A + \set{c} $.

\subsection{Sup and inequalities}
\begin{theorem}
    If $ A \subseteq B $, then $ \sup A \leqslant \sup B $.
\end{theorem}

\begin{definition}[Set comparison]
    Given two sets $ A $ and $ B $. If for every $ a \in A $ and every $ b \in B $, we have $ a \leqslant b $. Then we denote that $ A \leqslant B $. If $ A $ is bounded above by a number $ c $, that is, for every $ a \in A $ we have $ a \leqslant c $, then we denote that $ A \leqslant c $.
\end{definition}

$ A \leqslant c $ can be seen as a special case that $ A \leqslant \set{c} $.

\begin{theorem}[Comparison property] We can take supremum on both sides.
    \begin{enumerate}
        \item If $ A \leqslant B $, then $ \sup A \leqslant \sup B $
        \item If $ A \leqslant c $, then $ \sup A \leqslant c $
    \end{enumerate}
\end{theorem}


\subsection{Sup and functions}
Supremum is often applied to a function's range. In this circumstance, we have the following alternative notation of supremum, in which $ \sup $ can be seen as an operator:
\[ 
    \sup_{x \in S} f(x) \coloneq \sup \set{f(x) \colon x \in S} \,.
\]

We ignore the domain and write $ \sup f $, when it is clear from the context.

Relations bewteen functions are defined pointwise. For $ f, g \colon S \to \R $, the relation $ f \leqslant g $ means $ f(x) \leqslant g(x) $ for all $ x \in S $. Arithmetic operations also perform in pointwise manner. For example $ (f + g)(x) = f(x) + g(x) $.


\begin{theorem}
    If $ f \leqslant g $, then we can take supremum of both sides, that is
    \[ 
        \sup_{x \in S} f(x) \leqslant \sup_{x \in S} g(x) \,.
    \]
\end{theorem}

\begin{proof}
    For every $ x \in S $, $ f(x) \leqslant g(x) \leqslant \sup g $, this means that $ \sup g $ is an upper bound of $ \set{f(x) \colon x \in S} $, therefore $ \sup f \leqslant \sup g $.
\end{proof}

As a special case, if $ f \leqslant c $, then $ \sup f \leqslant c $.


\begin{theorem}
    $ \sup (f + g) \leqslant \sup f + \sup g $.
\end{theorem}

\begin{proof}
    Note that $ \set{f(x) + g(x) \colon x \in S} \subseteq \set{f(x) \colon x \in S} + \set{g(x) \colon x \in S} $.
\end{proof}


\end{document}