\documentclass{article}

\usepackage{parskip}
    \setlength{\parindent}{0em}
\usepackage{geometry}
    \geometry{left=4cm,right=4cm,top=2cm,bottom=2cm}
\usepackage{amsmath, amssymb, amsthm, mathtools}
\usepackage{thmtools}
    \renewcommand\qedsymbol{$\blacksquare$}
    \declaretheorem[numberwithin=section]{proposition}
    \declaretheorem[numberwithin=section]{theorem}
    \declaretheorem[numberwithin=section]{definition}
    \declaretheorem[numbered=no]{example}
    \renewcommand*{\proofname}{Proof}
\usepackage{caption}
\usepackage{xcolor}
\usepackage{graphicx}
\usepackage{float}
\usepackage{setspace} 	 % 行间距 \begin{spacing}{arg}
\usepackage{extarrows}
\usepackage{esint}
\usepackage{hyperref}
    \hypersetup{colorlinks=true,linktoc=all,linkcolor=blue}
\usepackage[italic=true]{derivative}

\renewcommand{\vec}[1]{\boldsymbol{\mathbf{#1}}}
\newcommand{\unit}[1]{\boldsymbol{\mathbf{\hat{#1}}}}
\renewcommand{\rm}{\mathrm}
\renewcommand{\cal}{\mathcal}
\newcommand{\scr}{\mathscr}
\newcommand{\E}{\mathrm e}
\newcommand{\R}{\mathbb R}
\newcommand{\Z}{\mathbb Z}
\newcommand{\N}{\mathbb N}
\newcommand{\Q}{\mathbb Q}
\newcommand{\C}{\mathbb C}
\DeclarePairedDelimiter\set{\lbrace}{\rbrace}
\DeclarePairedDelimiter\norm{\lVert}{\rVert}
\newcommand{\del}{\nabla}

\pagestyle{empty}

\begin{document}
\section{Derivatives}
Fermat's theorem gives us relation between interior extrema and zero derivatives.
\begin{theorem}[Fermat's theorem]
    If $ f : (a, b) \to \R $ attains local extremum at some interior point $ c \in (a, b) $ and $ f $ has derivative at $ c $, then $ f'(c) $ must be zero.
\end{theorem}

\begin{proof}
    Assume $ c $ is maximum. If $ f'(c) > 0 \text{ or } \infty $, for every $ x $ in some $ (c, c + \delta) $ we have $ f(x) > f(c) $ [contradiction!]. If $ f'(c) < 0 \text{ or } -\infty $, for every $ x $ in some $ (c - \delta, c) $ we have $ f(x) > f(c) $ [contradiction!]. So $ f'(c) = 0 $. Proof is similar when $ c $ is minimum.
\end{proof}

\subsection{Mean value theorem}
\begin{theorem}[Rolle's theorem]
    Assume real-valued function $ f \in C[a, b] \cap D(a, b) $, and $ f(a) = f(b) $, then there is $ c \in (a, b) $ such that $ f'(c) = 0 $.
\end{theorem}

\begin{proof}
    Since $ f $ is continuous on compact interval $ [a, b] $, we know $ f $ attains maximum and minimum values on $ [a, b] $. If either of them is interior point of $ [a, b] $, then by Fermat's theorem the proof is done. If both of them are end points, then $ f $ is constant since $ f(a) = f(b) $.
\end{proof}

We use Rolle's theorem to prove mean value theorem (MVT).
\begin{theorem}[Cauchy's mean value theorem]
    Given $ f, g \in C[a, b] \cap D(a, b) $, there is $ c \in (a, b) $ such that
    \begin{align} \label{eq:cmvt}
        \big[ f(b) - f(a) \big] g'(c) = \big[ g(b) - g(a) \big] f'(c)  \,.
    \end{align}
\end{theorem}
\begin{proof}
    Let $ h(x) = \big[ f(b) - f(a) \big] g(x) - \big[ g(b) - g(a) \big] f(x) $, then \eqref{eq:cmvt} is equivalent to $ h'(c) = 0 $. The idea is to use Rolle's theorem on $ h $. Note that $ h(b) - h(a) = 0 $, and indeed $ h \in C[a, b] \cap D(a, b) $ since $ f $ and $ g $ do.
\end{proof}

Let $ g $ to be $ g(x) = x $ and we obtain the usual MVT. MVT and CMVT can be considered as extensions of Rolle's theorem.

\subsection{Taylor's theorem}
\begin{theorem}[Taylor's theorem] \label{thm:taylor}
    Assume $ f \in D^{n} (\alpha, \beta) \cap C^{n - 1} [\alpha, \beta] $ and $ a \in [\alpha, \beta] $. For $ x \in [\alpha, \beta] $ and $ x \neq a $, there is $ c \in (a, x) \text { or } (x, a) $ such that
    \[
        f(x) = \sum_{k = 0}^{n-1} \dfrac{f^{(k)} (a)}{k!} (x - a)^k + \dfrac{f^{(n)}(c)}{n!} (x - a)^n \,.
    \]
\end{theorem}

We prove Taylor's theorem by the following extended theorem.

\begin{theorem}[Taylor's theorem extended]
    Assume $ f, g \in D^{n} (\alpha, \beta) \cap C^{n - 1} [\alpha, \beta] $ and $ a \in [\alpha, \beta] $. For $ x \in [\alpha, \beta] $ and $ x \neq a $, there is $ c \in (a, x) \text { or } (x, a) $ such that
    \begin{align} \label{eq:taylor-extended}
        \left[ f(x) - \sum_{k=0}^{n-1} \dfrac{f^{(k)} (a)}{k!} (x - a)^k \right] g^{(n)}(c) = \left[ g(x) - \sum_{k=0}^{n-1} \dfrac{g^{(k)} (a)}{k!} (x - a)^k \right] f^{(n)}(c) \,. 
    \end{align}
\end{theorem}
The key idea is to use CMVT. In order to do so, we have to make \eqref{eq:taylor-extended} to match the form $ [F(*) - F(*)] G'(\cdot) = [G(*) - G(*)] F'(\cdot) $. Consider the Taylor polynomial:
\[
    f(a) + f'(a) (x - a) + \dfrac{f^{(2)}(a)}{2!} (x - a)^2 + \cdots + \dfrac{f^{(n-1)}(a)}{(n-1)!} (x - a)^{n-1} \,.
\]
If we keep to think $ x $ as the variable, this polynomial will never produce $ f(x) $. Instead if we consider the expansion point $ a $ as variable, then expansion at $ x $ will give us $ f(x) $, and expansion at $ a $ will give us the summation part in the square bracket.

\begin{proof}
    Without loss of generality, assume $ x > a $. Let $ \displaystyle F(t) = \sum_{k=0}^{n-1} \dfrac{f^{(k)} (t)}{k!} (x - t)^{k} $ and $ \displaystyle G(t) = \sum_{k=0}^{n-1} \dfrac{g^{(k)} (t)}{k!} (x - t)^{k} $. Since $ f, g \in D^{n} (\alpha, \beta) \cap C^{n - 1} [\alpha, \beta] $, we have $ F, G \in C^{n - 1} [a, x] \cap D^{n} (a, x) $. So apply CMVT to $ F $ and $ G $, there is some $ c \in (a, x) $ such that
    \begin{align} \label{eq:etaylor1}
        [F(x) - F(a)] G'(c) = [G(x) - G(a)] F'(c) \,. 
    \end{align}
    Now $ F(x) = f(x) $, $ G(x) = g(x) $, and we have
    \[ 
        \begin{array}{ll} \displaystyle
            F(a) = \sum_{k = 0}^{n-1} \dfrac{f^{(k)} (a)}{k!} (x - a)^k \,, &
            G(a) = \sum_{k = 0}^{n-1} \dfrac{g^{(k)} (a)}{k!} (x - a)^k \,,  \\[1.5em]
            F'(a) = \dfrac{f^{(n)}(a)}{(n - 1)!} (x - a)^{n - 1} \,,       &
            G'(a) = \dfrac{g^{(n)}(a)}{(n - 1)!} (x - a)^{n - 1} \,.
        \end{array}
    \]
    Substitute into \eqref{eq:etaylor1}, we obtain \eqref{eq:taylor-extended}.
\end{proof}

By taking $ g(x) = (x - a)^n $, we get the usual Taylor's theorem (\ref{thm:taylor}) with Lagrange form of remainder.


\end{document}