\documentclass[UTF8]{ctexart}
\hfuzz=4pt
\usepackage{anyfontsize}
\usepackage{geometry}
    \geometry{left=4cm,right=4cm,top=2cm,bottom=2cm}
\usepackage{parskip}
    \setlength{\parindent}{0em}
    \setlength{\parskip}{1em}
\usepackage{amsmath, amssymb, amsthm, mathtools}
    \renewcommand*{\proofname}{proof}
\usepackage{thmtools}
    \renewcommand\qedsymbol{$\blacksquare$}
    \declaretheorem[numberwithin=section,shaded={rulecolor=cyan,rulewidth=2pt,bgcolor=white}]{definition}
    \declaretheorem[numberwithin=section,shaded={rulecolor=orange,rulewidth=2pt, bgcolor=white}]{theorem} 
    \newtheoremstyle{mystyle}{1em plus .2 em minus .2em}{1em plus .2 em minus .2em}{}{}{\bfseries}{.}{.5em}{}
    \theoremstyle{mystyle}
    \newtheorem{axiom}{Axiom}[section]
    \newtheorem{lemma}{Lemma}[section]
    \newtheorem{proposition}{Proposition}[section]
    \newtheoremstyle{myremark}{1em plus .2 em minus .2em}{1em plus .2 em minus .2em}{}{}{\itshape}{.}{.5em}{}
    \theoremstyle{myremark}
    \newtheorem*{remark}{Remark}
    \theoremstyle{plain}
    \newtheorem{corollary}{Corollary}[section]
\usepackage{xcolor} 	 % 颜色
\usepackage{graphicx} 	 % 引用图片
\usepackage{float}
\usepackage{centernot}
\usepackage{esvect} 	 % 向量箭头 \vv{} 
\usepackage{esint} 	 % 积分符号
\usepackage{mathrsfs}


% font
\newcommand{\ve}[1]{\boldsymbol{\mathbf{#1}}}
\newcommand{\unit}[1]{\boldsymbol{\mathbf{\hat{#1}}}}
\newcommand{\mr}{\mathrm}
\newcommand{\mcal}{\mathcal}
\newcommand{\mscr}{\mathscr}
% common symbol
\newcommand{\E}{\mathrm e}
\renewcommand{\I}{\mathrm i}
\newcommand{\R}{\mathbb R}
\newcommand{\Z}{\mathbb Z}
\newcommand{\N}{\mathbb N}
\newcommand{\Q}{\mathbb Q}
\newcommand{\C}{\mathbb C}
% matrix
\newcommand{\transp}{^{\top}}
\DeclareMathOperator{\tr}{tr}
\DeclareMathOperator{\GL}{GL}
\DeclarePairedDelimiter\set{\{}{\}}


\pagestyle{empty}

\begin{document}
\section{Group}
\subsection{Law of Composition}
\begin{definition}[\text{Composition}]
    Composition (or Law of composition) on a set S is to combine two element $ a, b \in S $, to get another element $ p $ in $ S $: \[ S \times S \to S \,.\] Here, $ \times $ means Cartesian product of two sets. We can denote the composition in several ways: \[ p = a b \quad p = a \cdot b \quad p = a \circ b \quad p = a + b \,.\]
\end{definition}

We will often use $ ab $ (or $ a \cdot b $ when neccessary) to denote the composition of $ a $ and $ b $ in this document.

\paragraph{Example}
\begin{itemize}
    \item In the set $ \N $, operation ``add'' $ + $ is a law of composition. It takes two elements of $ a, b \in \N $ and gives an element $ a + b \in \N $. e.g.\ $ (2, 3) \mapsto 5 $, $ (5, 1) \mapsto 6 $
    \item In the set $ \R $, operation ``multiply'' $ \cdot $ is a law of composition. It takes two elements of $ a, b \in \R $ and gives an element $ a \cdot b \in \R $. e.g.\ $ (-1, 4) \mapsto -4 $, $ (2, 3.5) \mapsto 7 $
\end{itemize}

Note that the definition of composition natrually brings out the property of closure --- the composition of two element of $ S $ is still in the same set.

A way of defining composition is using functions. $ f \colon S \times S \to S $, so for $ a, b \in S $, $ f(a, b) $ is the composition of $ a $ and $ b $.

\begin{definition}[\text{Associativity}]
    For element $ a $, $ b $ and $ c $, if the composition satisfies $ (ab)c = a(bc) $, then the composition is \textbf{associative}.
\end{definition}

For multiple element $ a_1 $, $ a_2 $, $ \dots $, $ a_n $, there's only one distinct way to define the composition of them:
\[ a_1 a_2 \cdots a_n = (a_1 \cdots a_i) (a_i \cdots a_n) \,,\]

where $ 1 \leqslant i < n $. For instance,
\[ a_1 a_2 a_3 a_4 = a_1 (a_2 a_3 a_4) = (a_1 a_2) (a_3 a_4) = (a_1 a_2 a_3) a_4 = a_1 (a_2 a_3) a_4 \,.\]

\begin{definition}[\text{Commutativity}]
    The composition of two element $ a $ and $ b $ is called \textbf{commutative} if $ ab = ba $.
\end{definition}

\paragraph{Example}
Addtion in $ \R $ is commutative: $ a, b \in \R $, $ a + b = b + a $.

\subsection{Special elements}
\begin{definition}[\text{Identity element}]
    If $ \forall s \in S $, $ \exists e \in S $ such that $ es = s $, then $ e $ is the \textbf{left identity} of $ S $. Likewise, $ e $ is the \textbf{right identity} if $ se = s $. If $ e $ is both left identity and right identity, then it's called a \textbf{two-sided identity} or simply \textbf{identity}.
\end{definition}

If we use multiplication to represent composition, then $ 1 $ is commonly used as the symbol of identity. And $ 0 $ is often identity for addition representation.

\paragraph{Example}
\begin{itemize}
    \item Concider zero in $ \Z $. For all $ a \in \Z $, $ 0 + a = a $, so $ 0 $ is the left identity. And by commutativity we also have $ a + 0 = a $, so $ 0 $ is also the right identity. Therefore, $ 0 $ is the identity of addition on $ \Z $
    \item $ 1 $ is the identity of multiplication on $ \R $, because $ \forall a \in Z $, $ 1 \cdot a = a \cdot 1 = a $
\end{itemize}

\begin{definition}[\text{Inverse}]
    Let $ 1 $ be the identity. If $ \forall a \in S $, $ \exists l \in S $ such that $ l a = 1 $, then $ l $ is called the \textbf{left inverse} of $ a $. Likewise, $ a r = 1 $ then $ r $ is called the \textbf{right inverse} of $ a $. If $ b $ is both left and right inverse of $ a $, then it's called the \textbf{two-sided inverse} or simply \textbf{inverse} of $ a $, denoted by $ a^{-1} $.
\end{definition}

\paragraph{Example}
\begin{itemize}
    \item $ -3 $ is the additive inverse of $ 3 $ in $ \R $, because $ (-3) + 3 = 3 + (-3) = 0 $ and $ 0 $ is the identity of addition.
    \item $ 1/2 $ is the multiplicative inverse of $ 2 $ in $ \R $, because $ (1/2) \times 2 = 2 \times (1/2) = 1 $ and $ 1 $ is the identity of multiplication.
\end{itemize}

A fraction $ \dfrac{a}{b} $ is exactly the composition of $ a $ and $ b^{-1} $. And the notation $ \dfrac{a}{b} $ is not recommended, because sometimes the composition is not commutative, therefore $ a b^{-1} $ and $ b^{-1} a $ are different.

The notations like $ a^n $ or $ a^{-n} $, $ n \in \N $ can be recursively defined as below:
\[ a^{n + 1} \coloneqq a^n a  \,,\]
\[ a^{-n - 1} = a^{-n} a^{-1} \,,\] 
\[ a^0 = 1 \,.\]

\begin{proposition}
    $ (a b)^{-1} = b^{-1} a^{-1} $.
\end{proposition}

\begin{proof}
    $ (ab)^{-1} (ab) = 1 $ is true, multiply $ b^{-1} $ on the right for both sides. $ (a b)^{-1} (a b) b^{-1} = 1 \cdot b^{-1} $, which is $ (a b)^{-1} a (b b^{-1}) = (a b)^{-1} a \cdot 1 = b^{-1} $. This time multiply $ a^{-1} $ on the right for both sides: $ (a b)^{-1} a a^{-1} = b^{-1} a^{-1} $, the left-hand side is exactly $ (a b)^{-1} $.
\end{proof}

And this can be easily generalized to $ n $ elements (using associativity and induction):
\[ (a_1 a_2 \dots a_n)^{-1} = a_n^{-1} \dots a_2^{-1} a_1^{-1} \,.\]

\subsection{Group}
\begin{definition}
    A group $ (G, \cdot) $ is a set $ G $ equipped with a binary operation $ \cdot $ which follows four axioms, namely \textbf{closure}, \textbf{associativity}, \textbf{identity} and \textbf{invertibility}.
\end{definition}    

\begin{remark}
    If a group is commutative, then it's called \textbf{abelian group}.
\end{remark}

The four axioms are explained below:
\paragraph{closure}
For all $ a, b $ in $ G $, the result of operation $ \cdot $ is still in $ G $. This can be written in the form: $ \forall a, b \in G, a \cdot b \in G $.

\paragraph{associativity}
$ \forall a, b, c \in G $, $ (a \cdot b) \cdot c = a \cdot (b \cdot c) $.

\paragraph{identity}
$ \exists\ e \in G $ such that, $ \forall a \in G $, the equation $ e \cdot a = a \cdot e = a $ holds. Such an element is unique and is called the \textbf{identity element}.

\paragraph{invertibility}
For each $ a \in G $, $ \exists\ b $ in $ G $, commonly denoted $ a^{-1} $, such that $ a \cdot b = b \cdot a = e $, where $ e $ is the identity element.

We use ordered pair to denote $ (G, \cdot) $ a set $ G $ equipped with operation $ \cdot $. So the two parts --- set and its operation --- together forms the algebraic structure. This is critical, because strictly speaking, a set on its own can not be a group. But informally, it's common to say that a set $ G $ is a group, if no ambiguity is caused.

\paragraph{Example}
These are some familiar abelian groups: $ (\Z, +) $, $ (\R, +) $, $ (\R^+, \times) $, $ (\C, +) $, $ (\C, \times) $. Take $ (\R^+, +) $ for example. 

\begin{enumerate}
    \item Two the addition of positive real numbers $ a, b $ is still a real number (closure)
    \item $ (a + b) + c = a + (b + c) $, which is associativity
    \item For any given $ a \in \R^+ $, $ 1 \times a = a \times 1 = a $ holds, which indicates that $ 1 $ is the multiplicative identity of $ \R^+ $
    \item For any given $ a \in \R^+ $, $ \exists a^{-1} $ such that $ a^{-1} \times a = a \times a^{-1} = 1 $ holds, which indicates all $ a \in \R^+ $ is invertable
\end{enumerate}

Therefore, $ (\R^+, \times) $ is a group. Also, for any positive real number $ a $ and $ b $, $ a \times b = b \times a $. So $ (\R^+, \times) $ is also an abelian group.


\begin{remark}
    Note that $ (\R, \times) $ is not a group, because $ 0 $ is not invertable: $ \centernot\exists r \in \R $ such that $ r \times 0 = 0 \times r = 1 $.
\end{remark}

Since groups are sets equipped with operations, and we have cardinality to describe how many elements we have in a set, it's natural to have a similar concept to describe the number of elements contained in a group.

\begin{definition}[\text{Order of a group}]
    The order of a group describe the number of elements contained in this group. Suppose we have group $ (G, \cdot) $, the order of this group equals the cardinality of $ G $, denoted by $ |G| $.
\end{definition}

\paragraph{Example}
The previous example, abelian group $ (\Z, +) $ is an infinite group, because $ \Z $ is an infinite set.

Because of invertibility property, a group has follows \textbf{cancellation law}.

\begin{proposition}
    Let $ a $, $ b $, $ c $ be elements of a group $ G $: 
    \begin{itemize}
        \item if $ ac = bc $ or $ ca = cb $ then $ a = b $
        \item if $ ac = c $ or $ ca = c $ then $ a = 1 $
    \end{itemize}
\end{proposition}

\begin{proof}
    Proofs of all cases are analogous --- by multiplying $ c^{-1} $ to both sides.
\end{proof}

Concider matrices. Not all matrices are invertable, so we can't just say matrix with multiplication operation is or is not a group.

\begin{definition}[\text{General linear group}]
    The general linear group of degree $ n $ is the set of $ n \times n $ invertable matrices:
    \[ \GL_n \coloneqq \set{n \times n \text{ invertable matrices}} \,.\]

    And enable to distinguish what kind of elements we are having in the matrices, notations like $ \GL_n(\R) $ or $ \GL_n(\C) $ are used.
\end{definition}

\subsection{半群}
半群是弱于群的概念.
\begin{definition}[\text{半群}]
    $ (G, \cdot) $ 被称为半群, 当且仅当 $ G $ 对 $ \cdot $ 封闭且 $ \cdot $ 满足结合律.
\end{definition}

如果 $ (G, \cdot) $ 中存在 $ a $, 满足 $ a a = a $. 则称 $ a $ 为 $ \cdot $ 运算的幂等元. 借助下面的引理可以证明, 有限的半群中必然存在幂等元.

\begin{lemma}
    如果对于有限半群 $ G $ 的元素 $ a $, 存在正整数 $ k \geqslant 2 $, 满足 $ a^k = a $, 则 $ G $ 中存在幂等元.
\end{lemma}

\begin{proof}
    对于 $ a^k = a $, 若 $ k = 2 $, $ a $ 为幂等元, 引理得证. 若 $ k > 2 $, 则将等式两边同时乘以 $ a^{k - 2} $. 得到 $ a^{2(k - 1)} = a^{k - 1} $. 即 $ \left( a^{k - 1} \right)^2 = a^{k - 1} $, 而 $ a^{k - 1} \in G $, 所以 $ G $ 中存在幂等元 $ a^{k - 1} $.
\end{proof}

\begin{proposition}
    有限的半群必然包含幂等元, 即若 $ G $ 为有限的半群, 则存在 $ a \in G $, 使得 $ aa = a $.
\end{proposition}

\begin{proof}
    对于任意 $ a \in G $, 考虑无限序列 \[ \left( a^{2^p} \right)_{p = 0}^\infty \colon \quad a, a^2, a^4, a^8, a^{16}, \dots \]

    由于封闭性, 序列中每一项都在 $ G $ 中, 于是必然存在不同的 $ s $, $ t $ 满足 $ a^{\displaystyle 2^s} = a^{\displaystyle 2^t} $. 因为如果不然, 序列中的每一项互不相同, 则 $ G $ 不可能有限. 不失一般性地假设 $ s > t $, 于是有: \[ a^{\displaystyle 2^{s}} = a^{\displaystyle 2^{t + (s - t)}} = a^{\displaystyle 2^{t} 2^{\displaystyle s - t}} = a^{\displaystyle 2^t} \,,\]

    于是得到 $ \left( a^{\displaystyle 2^t} \right)^{\displaystyle 2^{s - t}} = a^{\displaystyle 2^t} $. 于是我们找到了 $ b = a^{\displaystyle 2^t} \in G $, 使得存在 $ k = 2^{s - t} $, 满足 $ b^k = b $, 根据上一个引理, $ G $ 中存在幂等元.
\end{proof}









\end{document}