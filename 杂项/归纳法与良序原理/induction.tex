\documentclass[UTF8]{ctexart}

\usepackage{parskip}
    \setlength{\parindent}{0em}
    \setlength{\parskip}{1em}
\usepackage{geometry}
    \geometry{left=4cm,right=4cm,top=2cm,bottom=2cm}
\usepackage{amsmath, amssymb, amsthm, mathtools}
\usepackage{thmtools}
    \newtheorem{axiom}{Axiom}
    \newtheorem{lemma}{Lemma}
\pagestyle{empty}

\newcommand{\ve}[1]{\boldsymbol{\mathbf{#1}}}
\newcommand{\unit}[1]{\boldsymbol{\mathbf{\hat{#1}}}}
\renewcommand{\r}{\mathrm}
\renewcommand{\cal}{\mathcal}
\newcommand{\scr}{\mathscr}
\newcommand{\E}{\mathrm e}
\renewcommand{\I}{\mathrm i}
\newcommand{\R}{\mathbb R}
\newcommand{\Z}{\mathbb Z}
\newcommand{\N}{\mathbb N}
\newcommand{\Q}{\mathbb Q}
\DeclarePairedDelimiter\set{\{}{\}}

\title{数学归纳法与良序原理}


\begin{document}
\maketitle

\paragraph{数学归纳法} 
设有命题 $ P(n) $. 若满足下面两个条件:
\begin{enumerate}
    \item 基础情形: $ P(0) $ 为真
    \item 归纳假设: 若 $ P(n) $ 为真, 则 $ P(n + 1) $ 也为真
\end{enumerate}
则 $ P(n) $ 对于任意 $ n \in \N $ 均为真.


\paragraph{强归纳法(完全归纳法)}
设有命题 $ P(n) $. 若能够证明对于任意自然数 $ m $ 都有: 对任意 $ k < m $, $ P(k) $ 为真, 则 $ P(m) $ 为真. 那么 $ P(n) $ 对于任意 $ n \in \N $ 均为真.

首先, 注意此处的强归纳法无需显式地说明基础情形 $ P(0) $ 成立, 因为取 $ m = 0 $, 不存在 $ k < m $, 那么 $ P(k) $ 空真, 于是根据条件 $ P(m) = P(0) $ 真, 也就得到了基础情形. 这是我们在条件中使用 $ k < m $ 的好处. 当然, 将条件阐述为``对任意 $ k \leqslant m $ ...'' 也不是不行, 只是需要阐明 $ P(0) $ 为真, 因为条件不再蕴含这一信息.

其次, 以上两个归纳法的起点都是 $ 0 $, 实际上归纳法可以更灵活. 下面用更加精简的记号重新推广上面的归纳法. 注: $ P(n) $ 默认表示 $ P(n) $ 为真.

\vskip 1em
\begin{lemma}[\text{数学归纳法(不完全归纳法)}]
    若 $ P(n) $ 满足:
    \begin{enumerate}
        \item 基础情形: $ P(m_0) $
        \item 归纳假设: 对任意 $ n \geqslant m_0 $, $ P(n) \Longrightarrow P(n + 1) $
    \end{enumerate}
    则 $ \forall n \geqslant m_0 $ 均为真.
\end{lemma}

\vskip 1em
\begin{lemma}[\text{强归纳法(完全归纳法)}]
    若对于任意 $ m \geqslant m_0 $, 都满足: 对任意 $ m_0 \leqslant k < m $, $ P(k) $ 成立, 那么 $ P(m) $ 成立. 则 $ P(n) $ 对任意 $ n \geqslant m_0 $ 均成立. 换句话说: \[ \forall m \geqslant m_0 \Bigl[ \Bigl( P(m_0) \land P(m_0 + 1) \land \cdots \land P(m - 1) \Bigr) \Longrightarrow P(m) \Bigr] \Longrightarrow (\forall n \geqslant m_0) P(n) \,.\]
\end{lemma}

也可以换种表述, 使之于数学归纳法对应:
\begin{lemma}[\text{强归纳法(等价表述)}]
    若 $ P(n) $ 满足:
    \begin{enumerate}
        \item 基础情形: $ P(m_0) $
        \item 归纳假设: 对任意 $ m \geqslant m_0 $, $ P(m_0) \land P(m_0 + 1) \land \cdots \land P(m) \Longrightarrow P(m + 1) $
    \end{enumerate}
    则 $ \forall n \geqslant m_0 $, $ P(n) $ 为真
\end{lemma}

下面说明数学归纳法和强归纳法等价, 我们可以说明两者可以相互推导.
\paragraph{强归纳法到数学归纳法}
注意到, 如果对于任意 $ m \geqslant m_0 $: \[ \Bigl( P(m_0) \land P(m_0 + 1) \land \cdots \land P(m - 1) \Bigr) \Longrightarrow P(m) \] 设 $ Q(n) $ 表示 $ P(m_0) \land P(m_0 + 1) \land \cdots \land P(n - 1) $, 即 $ \forall m_0 \leqslant k < m $, $ P(k) $ 成立. 

于是根据条件有任意的 $ m \geqslant m_0 $: $ Q(m) \Longrightarrow P(m) $. 只需证明 $ Q(n) $ 对所有 $ n \geqslant m_0 $ 成立, 即可证明 $ P(n) $ 对 $ n \geqslant m_0 $ 成立. 这便转化到了数学归纳法上:
\begin{itemize}
    \item $ Q(m_0) $ 成立, 因为 $ \forall m_0 \leqslant k < m_0, P(k) $ , 这是空真的命题
    \item $ Q(n) $ 成立, 则 $ P(n) $ 成立, 于是 $ Q(n) \land P(n) = Q(n + 1) $ 成立
\end{itemize}


\end{document}