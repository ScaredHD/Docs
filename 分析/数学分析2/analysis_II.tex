\documentclass[UTF8]{ctexart}
\hfuzz=4pt

\usepackage{parskip}
    \setlength{\parindent}{0em}
    \setlength{\parskip}{1em}
\usepackage{geometry}
    \geometry{left=4cm,right=4cm,top=2cm,bottom=2cm}
\usepackage{amsmath, amssymb, amsthm, mathtools}
\usepackage{thmtools}
    \renewcommand\qedsymbol{$\blacksquare$}
    \declaretheorem[numberwithin=section,shaded={rulecolor=cyan,rulewidth=2pt,bgcolor=white}]{definition}
    \declaretheorem[numberwithin=section,shaded={rulecolor=orange,rulewidth=2pt, bgcolor=white}]{theorem} 
    \newtheoremstyle{mystyle}{1em plus .2 em minus .2em}{1em plus .2 em minus .2em}{}{}{\bfseries}{.}{.5em}{}
    \theoremstyle{mystyle}
    \newtheorem{axiom}{Axiom}[section]
    \newtheorem{lemma}{Lemma}[section]
    \newtheorem{proposition}{Proposition}[section]
    \newtheoremstyle{myremark}{1em plus .2 em minus .2em}{1em plus .2 em minus .2em}{}{}{\itshape}{.}{.5em}{}
    \theoremstyle{myremark}
    \newtheorem*{remark}{Remark}
    \theoremstyle{plain}
    \newtheorem{corollary}{Corollary}[section]
\usepackage{graphicx}
\usepackage{xcolor}
\usepackage{float}
\usepackage{esint}
\usepackage{hyperref}
    \hypersetup{colorlinks=true,linktoc=all,linkcolor=blue}

\newcommand{\ve}[1]{\boldsymbol{\mathbf{#1}}}
\newcommand{\unit}[1]{\boldsymbol{\mathbf{\hat{#1}}}}
\renewcommand{\r}{\mathrm}
\renewcommand{\cal}{\mathcal}
\newcommand{\scr}{\mathscr}

\newcommand{\E}{\mathrm e}
\renewcommand{\I}{\mathrm i}
\newcommand{\R}{\mathbb R}
\newcommand{\Z}{\mathbb Z}
\newcommand{\N}{\mathbb N}
\newcommand{\Q}{\mathbb Q}
\renewcommand{\C}{\mathbb C}
\DeclarePairedDelimiter\set{\{}{\}}
\def \DD #1.#2.#3 {\dfrac{d^{#1} #2}{d #3^{#1}}}
\def \PP #1.#2.#3 {\dfrac{\partial^{#1} #2}{\partial #3^{#1}}}
\def \dd #1.#2 {\dfrac{d #1}{d #2}}
\def \pp #1.#2 {\dfrac{\partial #1}{\partial #2}} 
\newcommand{\del}{\nabla}
\DeclareMathOperator{\cl}{cl}

\pagestyle{empty}

\begin{document}
\tableofcontents
\section{实数轴上的连续函数}
\subsection{实直线}
区间的定义省略. 下面是关于区间的一些术语及概念:
\begin{itemize}
    \item 半无限区间: 一个端点是 $ -\infty $ 或 $ +\infty $ 的区间
    \item 双无限区间: 两个端点都是 $ -\infty $ 或 $ +\infty $ 的区间
    \item 有界区间: 不是无限区间. 意味着存在正实数 $ M $ 使得该区间是 $ [-M, M] $ 的子集.
\end{itemize}

退化区间:
\begin{itemize}
    \item 若 $ a > b $: $ (a, b) $, $ (a, b] $, $ [a, b) $ 和 $ [a, b] $ 都是空集
    \item 若 $ a = b $: $ (a, b) $, $ (a, b] $, $ [a, b) $ 是空集, 而 $ [a, b] $ 为单点集 $ \set{a} $
\end{itemize}

\paragraph{一个好用的记号}
本文档中定义整数区间如下:
\[ [a .. b] \coloneqq [a, b] \cap \Z \,.\]

即 $ [a, b] $ 区间内的整数. 同理可以定义以 $ .. $ 分隔的开区间和半开半闭区间. 该记号借自计算机科学界, 可以方便地表示一个范围内的整数. 如:
\[ [0 .. 3] = \set{0, 1, 2, 3} \,,\]
\[ [-2 .. 2] = \set{-2, -1, 0, 1, 2} \,,\]
\[ [0 .. 5) = \set{0, 1, 2, 3, 4} \,,\]
\[ (1 .. 3) = \set{2} \,,\]
\[ (2 .. 4] = \set{3, 4} \,.\]

注意按照我们交集的的定义方式, 区间的端点不是必须为整数. 比如下面的记号也是良定义的:
\[ [1.2 .. 3.9] = [1.2, 4.9] \cap \Z = \set{2, 3, 4} \,,\]
\[ (0.5 .. 2.5) = [0.5, 2.5) \cap \Z = \set{1, 2} \,.\]

只是一般不会也没有必要这样使用, 并且小数点混在其中太过难看.


\subsection{附着点与极限点}
\subsubsection{附着点}
\begin{definition}[\text{附着点 (Adherent point)}]
    对于 $ X \subseteq \R $, 称 $ x $ 为 $ X $ 的附着点, 当且仅当对任意实数 $ \epsilon > 0 $, 存在 $ y \in X $, 使得 $ |x - y| \leqslant \epsilon $.
\end{definition}

绝对值 $ |x - y| $ 的集合意义为 $ x $ 与 $ y $ 的距离, 既 $ d(x, y) $, 而 $ d(x, y) $ 小于任意给定正实数. 直观地说, 若 $ x $ 是 $ X $ 的附着点, 那么 $ x $ 无限靠近集合 $ X $. 此外, 若 $ x $ 本就是 $ X $ 中的元素, 那么 $ x $ 显然也是附着点.

\paragraph{例}
对于集合 $ (1, 2] \cup \set{3} $, $ 1 $, $ 2 $, $ 3 $ 都是其附着点, $ 1.5 $ 也是.

\paragraph{例}
对于集合 $ (1, 2] $, $ 0.5 $ 不是附着点. 因为对于 $ \epsilon = 0.1 $, 所有 $ (1, 2] $ 中的元素 $ y $ 与 $ 0.5 $ 的距离 $ |y - 0.5| $ 都大于 $ 0.1 $.

集合的所有附着点, 构成了这个集合的闭包. 直观地说: 就是扩大集合, 使其包含所有无限靠近原集合的点.

\begin{proposition}[\text{附着点的性质}] 设 $ X \subseteq \R $:
    \begin{enumerate}
        \item 若 $ x $ 为 $ X $ 的附着点, $ Y $ 为任意集合, 则 $ x $ 为 $ X \cup Y $ 的附着点
        \item 若 $ x $ 为 $ X \cap Y $ 的附着点, $ x $ 同时为 $ X $ 和 $ Y $ 的附着点
    \end{enumerate}
\end{proposition}

\begin{proof}
    按照定义即可.
\end{proof}

\begin{definition}[\text{闭包 (Closure)}]
    $ X $ 的所有附着点的集合称为 $ X $ 的闭包, 记作 $ \cl (X) $ 或 $ \overline X $.
\end{definition}

\begin{proposition}[\text{闭包算子的性质}] 设 $ X, Y \subseteq \R $:
    \begin{enumerate}
        \item $ X \subseteq \cl (X) $
        \item $ \cl (X \cup Y) = \cl (X) \cup \cl (Y) $
        \item $ \cl (X \cap Y) \subseteq \cl (X) \cap \cl (Y) $
        \item $ \cl (\cl (X)) = \cl (X) $
        \item $ X \subseteq Y $, 则 $ \cl (X) \subseteq \cl (Y) $
    \end{enumerate}
\end{proposition}

\begin{remark}
    $ \cl(X \cap Y) \subseteq \cl(X) \cap \cl(Y) $, 这个式子不能像并 $ \cup $ 情形的式子一样取等于. 因为可以找到反例: $ \cl \big( (0, 1) \cap (1, 2) \big) = \varnothing \neq \set{1} = \cl(0, 1) \cap \cl(1, 2) $.
\end{remark}

使用闭包的记号, 就可以将 $ x $ 附着于 $ X $ 记作: $ x \in \cl(X) $, 其满足的性质也可以记作:
\begin{itemize}
    \item $ x \in \cl(X) \implies x \in \cl(X \cup Y) $
    \item $ x \in \cl(X \cap Y) \implies x \in \cl(X) \land x \in \cl(Y) $
\end{itemize}

\begin{proposition}[\text{区间的闭包}]
    若 $ I $ 为 $ (a, b) $, $ (a, b] $, $ [a, b) $, $ [a, b] $ 中的一个, 那么 $ \cl (I) = [a, b] $.
\end{proposition}

下面以 $ (a, b) $ 为例, 其余证明同理.
\begin{proof}
    首先, 证明 $ [a, b] $ 中的任意点 $ x $ 都是 $ (a, b) $ 的附着点. 有三种情况: (1) $ x \in (a, b) $, 则 $ x $ 自然为 $ (a, b) $ 的附着点; (2) $ x = a $, 则 $ x $ 按照定义是附着于 $ (a, b) $ 的; (3) $ x = b $, $ x $ 也是附着于 $ (a, b) $ 的.

    接着证明 $ (a, b) $ 的所有附着点都位于 $ [a, b] $ 中. 设 $ x $ 为 $ (a, b) $ 的附着点, 但 $ x \notin [a, b] $. 于是有两种情况: (1) $ x < a $: $ \forall \epsilon > 0 $, $ \exists y \in (a, b) $, 使得 $ |x - y| \leqslant \epsilon $. 由于 $ a - x > 0 $, 取 $ \epsilon = a - x $, $ |x - y| \leqslant a - x $. 所以有 $ x - a \leqslant x - y $, 即 $ y \leqslant a $, 这与 $ y \in (a, b) $ 矛盾. (2) $ x > b $ 同理, 取 $ \epsilon = b - x $ 可以得到矛盾.

    上面两方面结合起来说明了所有 $ (a, b) $ 附着点构成的集合恰好是 $ [a, b] $.
\end{proof}

\begin{definition}[\text{闭集合}]
    $ X \subseteq \R $, 若 $ X = \cl(X) $ 则称集合 $ X $ 是闭的.
\end{definition}

\begin{proposition}
    $ \varnothing $, $ \N $, $ \Z $, $ \R $ 的闭包为自身, $ \Q $ 的闭包为 $ \R $. 所以 $ \Q $ 不是闭集合, 而 $ \varnothing $, $ \N $, $ \Z $, $ \R $ 是闭集合.
\end{proposition}

下面引理表明, 附着点可以由集合内的序列来逼近. 其证明用到选择公理.
\begin{lemma}
    $ X \subseteq \R $, 则 $ x $ 为 $ X $ 的附着点, 当且仅当存在一个收敛到 $ x $ 的序列 $ (x_n)_{n = 1}^\infty $, 且序列的每一项都是 $ X $ 中的值 ($ \forall n \geqslant 1 $, $ x_n \in X $).
\end{lemma}

\begin{proof}
    考虑集合 $ X_n \coloneqq \set{e \in X \colon x - 1/n \leqslant e \leqslant x + 1/n} $. 由于 $ x $ 是 $ X $ 的附着点, $ \forall \epsilon > 0 $, $ \exists y \in X $, $ |x - y| \leqslant \epsilon $. 于是 $ x - \epsilon \leqslant y \leqslant x + \epsilon $ 对于任意正实数 $ \epsilon $ 均成立. 那么对任意 $ n \geqslant 1 $, 只需取 $ \epsilon = 1/n $, 此时能够找到 $ e \in X $, 满足 $ x - 1/n \leqslant e \leqslant x + 1/n $, 这说明所有 $ X_n $, $ n \geqslant 1 $ 都是非空的. 根据选择公理, 我们可以依次从中选出一个元素, 构成 $ (x_1, x_2, \dots) $. 所以序列 $ (x_i)_{i = 1}^\infty $ 的每一项 $ x - 1/n \leqslant x_i \leqslant x + 1/n $, 根据夹逼定理, 序列 $ (x_i)_{i = 1}^\infty $ 收敛于 $ x $.
\end{proof}

通过此条引理, 可以将闭包用序列的语言来定义.

\begin{corollary}
    $ X \subseteq \R $ 且 $ X $ 是闭的 ($ \cl(X) = X $). 收敛序列 $ (x_n)_{n = 1}^\infty $ 的每一项 $ x_n $ 都是 $ X $ 中的元素, 则 $ (x_n)_{n = 1}^\infty $ 收敛到 $ X $ 中的一点. 反过来, 若 $ X $ 中的任意收敛序列的极限都是 $ X $ 中的元素, 则 $ X $ 是闭的.
\end{corollary}

\subsubsection{极限点}
下面是一个和附着点 (adherent point) 非常相似但又有着细微差别的概念---极限点 (limit point).

\begin{definition}[\text{极限点(limit point)}]
    $ X \subseteq \R $, $ x $ 是 $ X $ 的极限点, 当且仅当 $ x $ 是 $ X \setminus \set{x} $ 的附着点. 若 $ x \in X $, 但 $ x $ 不是 $ X \setminus \set{x} $ 的附着点, 即存在 $ \epsilon > 0 $, 对任意 $ y \in X \setminus \set{x} $ 都有 $ |x - y| > \epsilon $. 此时称 $ x $ 是 $ X $ 的孤立点, 简称孤点.
\end{definition}

也就是说, $ x $ 是 $ X $ 的附着点, 则 $ x $ 能被 $ X $ 中元素的序列 (包括 $ x $ 本身) 逼近. 而 $ x $ 是极限点, 则 $ x $ 能被 $ X $ 中不同于 $ x $ 的元素组成的序列 (即 $ X \setminus \set{x} $ 中的序列) 逼近. 

\begin{lemma}
    $ X \subseteq \R $, 则 $ x $ 为 $ X $ 的极限点, 当且仅当存在一个收敛到 $ x $ 的序列 $ (x_n)_{n = 1}^\infty $, 且序列的每一项都是 $ X \setminus \set{x} $ 中的值 ($ \forall n \geqslant 1 $, $ x_n \in X \setminus \set{x} $).
\end{lemma}

下面的命题说明了, 集合的附着点可以分为极限点和孤点互斥的两部分.
\begin{proposition}
    $ x $ 是 $ X $ 的附着点. 则 $ x $ 要么是 $ X $ 的极限点, 要么是 $ X $ 的孤点, 且不可同时成立.
\end{proposition}

\begin{proof}
    设 $ x $ 为 $ X $ 的附着点. 若 $ x $ 还是 $ X \setminus \set{x} $ 的附着点, 则 $ x $ 为极限点. 反之, $ x $ 为孤点.

    另一方面, 假设 $ x $ 为 $ X $ 的极限点. $ x $ 是 $ X \setminus \set{x} $ 的附着点, 那么 $ x $ 也是 $ X \setminus \set{x} \cup \set{x} = X $ 的附着点. 假设 $ x $ 为 $ X $ 的孤点, 按照定义 $ x \in X $, 于是 $ x $ 为 $ X $ 的附着点.
\end{proof}

\begin{proposition}[\text{区间与极限点}]
    若 $ I $ 为下面区间的任意一个: $ (a, b) $, $ (a, b] $, $ [a, b) $, $ [a, b] $, $ (a, +\infty) $, $ [a, +\infty) $, $ (-\infty, b) $, $ (-\infty, b] $; 那么 $ I $ 中每一个点都是 $ I $ 的极限点.
\end{proposition}

下面以 $ [a, b] $ 为例证明, 其余同理.

\begin{proof}
    考虑 $ x \in [a, b] = I $. (1) $ x = a $: 考虑序列 $ (x + 1/n)_{n = N}^\infty $, 当 $ N $ 充分大时, 序列落入区间 $ I \setminus \set{a} $ 中, 且这个序列收敛到 $ x $, 说明 $ x $ 为极限点. (2) $ x = b $ 同理, 将序列换为 $ (x - 1/n)_{n = N}^\infty $ 即可. (3) $ x \in (a, b) $: 上面的论述仍然有效, 如 $ (x + 1/n)_{n = N}^\infty $ 在 $ N $ 充分大时落入 $ (x, b) \subseteq I \setminus \set{a} $ 且收敛到 $ x $.
\end{proof}


\subsubsection{有界集}
\begin{definition}[\text{有界}]
    实直线的子集 $ X $ 是有界的, 当且仅当 $ \exists M > 0 $, $ X \subseteq [-M, M] $.
\end{definition}

\paragraph{例}
半无限区间和双无限区间都是无界的. 以 $ (a, +\infty) $ 为例: 假设其为有界的, 则存在正的 $ M \in \R $, $ (a, +\infty) \subseteq [-M, M] $. 也就是说对于任意 $ x \in (a, +\infty) $, $ x \in [-M, M] $: 若 $ M <= a $, $ M <= a < x $, 所以 $ x $ 不在 $ [-M, M] $ 中; 若 $ M > a $, 取 $ x = M + 1 > M > a $, $ x \in (a, +\infty) $ 而  $ x \notin [-M, M] $. 所以产生了矛盾, $ (a, +\infty) $ 不是有界的.

\begin{theorem}[\text{实直线上的 Heine-Borel 定理}]
    设 $ X \subseteq \R $, 则下面两个命题等价:
    \begin{itemize}
        \item $ X $ 是闭的且有界
        \item $ (a_n)_{n = 0}^\infty $ 是 $ X $ 中的序列, 则存在子列 $ (a_{n_j})_{j = 0}^\infty $, 该子列收敛到 $ X $ 中的某个数.
    \end{itemize}
\end{theorem}

\begin{remark}
    $ \R $ 中有界的闭集合称为紧致集 (compact set), 简称紧集.
\end{remark}

换句话说, 紧致集中的任意序列都有收敛到该集合中的子列.

要证明 Heine-Borel 定理, 要用到 Bolzano-Weirestrass 定理.

\begin{proof}
    设 $ X $ 是紧致集, $ (a_n)_{n = 0}^\infty $ 是 $ X $ 中的序列, 即 $ \forall n \in \N $, $ a_n \in X $. 由 Bolzano-Weirestrass 定理, 该序列一定存在收敛子列, 且收敛到 $ X $ 中的某个数.

    设 $ (a_n)_{n = 0}^\infty $ 是 $ X $ 中的任意序列, 且存在子列 $ (a_{n_j})_{j = 0}^\infty $ 收敛到 $ X $ 中的某个数 $ L $.
\end{proof}

\subsection{函数的极限}
\begin{definition}[\text{函数极限-I}]
    设 $ E \subseteq X \subseteq \R $, 定义函数 $ f \colon X \to \R $ 在 $ x_0 $ 处沿着 $ E $ 收敛到 $ L $, 当且仅当对任意实数 $ \epsilon > 0 $, 都存在 $ \delta > 0 $, 对于一切满足 $ |x - x_0| \leqslant \delta $ 的 $ x \in E $, 都有: \[ |f(x) - L| \leqslant \delta \,.\] 记作: \[ \lim_{x \to x_0; x \in E} f(x) = L \,.\]
\end{definition}

注意这里和一般极限定义的区别. $  $


\end{document}