\documentclass[UTF8]{ctexart}
\hfuzz=4pt

\usepackage{parskip}
    \setlength{\parindent}{0em}
\usepackage{geometry}
    \geometry{left=4cm,right=4cm,top=2cm,bottom=2cm}
\usepackage{amsmath, amssymb, amsthm, mathtools}
\usepackage{thmtools}
    \renewcommand\qedsymbol{$\blacksquare$}
\usepackage{setspace} 	 % 行间距 \begin{spacing}{arg}
\usepackage{hyperref}
    \hypersetup{colorlinks=true,linktoc=all,linkcolor=blue}
\usepackage{verbatim}

\newcommand{\E}{\mathrm e}
\renewcommand{\I}{\mathrm i}
\newcommand{\R}{\mathbb R}
\newcommand{\Z}{\mathbb Z}
\newcommand{\N}{\mathbb N}
\newcommand{\Q}{\mathbb Q}
\renewcommand{\C}{\mathbb C}
\DeclarePairedDelimiter\set{\{}{\}}
\DeclarePairedDelimiter\Floor{\lfloor}{\rfloor}
\DeclarePairedDelimiter\Ceil{\lceil}{\rceil}
\DeclareMathOperator{\truncate}{truncate}
\DeclareMathOperator{\floor}{floor}
\DeclareMathOperator{\ceil}{ceil}
\DeclareMathOperator{\sgn}{sgn}
\DeclareMathOperator{\Frac}{frac}

\title{取整函数}

\begin{document}
\maketitle
\tableofcontents

\section{取整函数}
\subsection{定义}
对于 $ x \in \R $, 常见有三种取整方式:
\begin{itemize}
    \item 向上取整 (round up, ceil): 取大于等于 $ x $ 的整数中最靠近 $ x $ 的, 也可以看作是朝 $ +\infty $ 的方向取整
    \[ \ceil(x) \equiv \lceil x \rceil \coloneqq \min \set{n \in \Z \colon n \geqslant x} \]
    \item 向下取整 (round down, floor): 取小于等于 $ x $ 的整数中最靠近 $ x $ 的, 也可以看作是朝 $ -\infty $ 的方向取整
    \[ \floor(x) \equiv \lfloor x \rfloor \coloneqq \max\set{n \in \Z \colon n \leqslant x} \]
    \item 向零取整 (truncate): 朝 $ 0 $ 的方向取离 $ x $ 最近的整数
    \[ \truncate(x) \coloneqq \sgn (x) \lfloor |x| \rfloor \]
\end{itemize}

\subsection{性质}
\subsubsection{不等式}
根据定义, 可以得到最基本的不等式:
\[ 
    x - 1 < \Floor{x} \leqslant x \leqslant \Ceil{x} < x + 1 \,.
\]
从上式也可以看到
\[ 
    \Floor{x} \leqslant \Ceil{x} \,.
\]
以及下面的事实:
\[ 
    \Ceil{x} - \Floor{x} = \begin{cases}
        0 & \text{if } x \in \Z \\
        1 & \text{if } x \notin \Z
    \end{cases}  \,.
\]
也就是说, 当 $ x \notin \Z $ 时, $ \Floor{x} + 1 = \Ceil{x} $, 此时 $ \Floor{x} < x < \Ceil{x} $; 而当 $ x \in \Z $ 时, $ \Floor{x} = x = \Ceil{x} $.

\subsection{加法}
设 $ x \in \R $, $ x \in \Z $, 则有:
\[ 
    \Floor{x + n} = \Floor{x} + n \,,
\]
\[ 
    \Ceil{x + n} = \Ceil{x} + n \,.
\]
\begin{proof} 
    下面以 $ \Floor{x + n} = \Floor{x} + n $ 为例, 其余同理.
    
    记 $ \Frac(x) $ 为 $ x $ 的小数部分. 显然对于任意正实数 $ x $, 都有 $ x = \Floor{x} + \Frac(x) $; 而对于负实数 $ x = \Ceil{x} + \Frac(x) $. 当 $ x \in \Z $ 时, $ \Frac(x) = 0 $, 而当 $ x \in \Z $ 时, $ 0 < \Frac(x) < 1 $.

    下面分两种情况讨论: (1) 当 $ x \in \Z $ 时, $ x + n \in \Z $, 此时根据基本不等式: $ \Floor{x + n} = x + n $, $ \Floor{x} = x $, 等式成立. (2) 当 $ x \notin \Z $ 时: 若 $ x > 0 $, $ \Floor{x + n} = \Floor{\Floor{x} + \Frac(x) + n} = \Floor{\Floor{x} + n + \Frac(x)} $. 此时 $ \Floor{x} + n $ 为整数且 $ \Floor{x} + n < \Floor{x} + n + \Frac(x) $. 通过反证法可以得到, $ (\Floor{x} + n , \Floor{x} + n + \Frac(x)] $ 间不存在整数, 故 $ \Floor{x} + n $ 为小于等于 $ x + n $ 的最大整数, 这就证明了等式. $ x < 0 $ 的情况可同理证明.
\end{proof}

\subsection{和不等式}
对于任意 $ x, y \in \R $, 有:
\[ 
    \Floor{x} + \Floor{y} \leqslant \Floor{x + y} \leqslant \Floor{x} + \Floor{y} + 1 \,,
\]
\[ 
    \Ceil{x} + \Ceil{y} - 1 \leqslant \Ceil{x + y} \leqslant \Ceil{x} + \Ceil{y} \,.
\]



\section{代码实现}
C++ 的整数除法默认情况会截断小数部分, 效果等同于朝零取整.
\begin{verbatim}
    int i = 5 / 2;  // i = 2
    int j = -5 / 2;  // j = -2
\end{verbatim}


\end{document}